%!TEX TS-program = xelatex
%! TEX encoding = UTF-8 Unicode

%========================================全文布局
\documentclass[twoside,openany]{book}
\usepackage[screen,paperheight=14.4cm,paperwidth=10.8cm,
left=2mm,right=2mm,top=2mm,bottom=5mm]{geometry}

\usepackage[]{microtype}
\usepackage{graphicx}
\usepackage{amssymb,amsmath}
\usepackage{booktabs}
\usepackage{titletoc}
\usepackage{titlesec}
\usepackage{tikz}
\usepackage{enumerate}
\usepackage{wallpaper}
\usepackage{indentfirst}
%========================================设置字体
\usepackage{ctex}
%\usepackage[CJKnumber]{xeCJK}
\usepackage{xpinyin}
\setCJKmainfont[BoldFont={Adobe Heiti Std R}]{Hiragino Sans GB W3}
\setCJKfamilyfont{kai}{Adobe Kaiti Std R}
\setCJKfamilyfont{hei}{Adobe Heiti Std R}
\setCJKfamilyfont{fsong}{Adobe Fangsong Std R}

\newcommand{\kai}[1]{{\CJKfamily{kai}#1}}
\newcommand{\hei}[1]{{\CJKfamily{hei}#1}}
\newcommand{\fsong}[1]{{\CJKfamily{fsong}#1}}

\renewcommand\contentsname{目~录~}
\renewcommand\listfigurename{图~列~表~}
\renewcommand\listtablename{表~目~录~}
\usepackage{romannum}
%========================================章节样式
\titlecontents{chapter}
[0em]
{}
{\large\CJKfamily{hei}{}}
{}{\dotfill\contentspage}%用点填充
%
\titlecontents{section}
[4em]
{}
{\thecontentslabel\quad}
{}{\titlerule*{.}\contentspage}

\titleformat{\chapter}[display]
	{\CJKfamily{fsong}\large\centering}
	{\titlerule[1pt]%
	 \filleft%
	}
	{-7ex}
	{\Huge
	 \filright}
	[{\titlerule[1pt]}]

%========================================设置目录
\usepackage[setpagesize=false,
            linkcolor=black,
            colorlinks, %注释掉此项则交叉引用为彩色边框(将colorlinks和pdfborder同时注释掉)
            pdfborder=001   %注释掉此项则交叉引用为彩色边框
            ]{hyperref}

\setlength{\parindent}{2em} %首行缩进
\linespread{1.2}              %行距
\setlength{\parskip}{15pt}    %段距

%========================================页眉页脚
\usepackage{fancyhdr}
\pagestyle{fancy}
\fancyhf{}
\fancyfoot{}
\fancyfoot[LE,RO]{\scriptsize\thepage}
\setlength{\footskip}{6pt}
%========================================标题作者
\title{牧羊少年奇幻之旅}
\author{[巴西]保罗·科埃略\\ 孙成敖\ 译}
\date{}
\newcommand{\mt}[1]{\textbullet \textbf{#1}}
%========================================正文
\begin{document}
\TileSquareWallPaper{1}{TGTamber}%背景图片
\pagenumbering{arabic}
\maketitle
\tableofcontents
%\newpage

\zihao{4}
\noindent
\chapter*{序言}\addcontentsline{toc}{chapter}{\large\CJKfamily{hei}序言}

《牧羊少年奇幻之旅》是一部象征性作品,它有别于非小说体的《朝圣》,这一说明至关重要。

我曾花费十一年时间研究炼金术。点铁成金或发现长命液的天真想法实在太迷人了,足以令人不去注意任何其他的魔术。我承认,长命液曾使我最感诱惑:在领悟和感受到上帝存在之前,万物终有一天将会消亡的想法使人感到绝望。因此,当得知有可能获取一种可以使我生命延长许多年的液体时,我便决定全身心地致力于它的提炼。

七十年代初是社会发生大变革的一个时期,那时还没有关于炼金术的严肃出版物间世。于是我像本书中的一个人物那样,开始用我所有的很少一点钱购买进口图书,并且每天花费许多时间致力于研究复杂难懂的符号学。我在里约热内卢曾找过两三位认真从事元精研究的人,可他们拒绝见我。我还认识许多自称是炼金家的人,他们拥有自己的实验室,并许诺教我点铁成金技艺的秘密,而我要给他们真正的财富作为交换条件。今天我已明白,他们其实根本不懂要准备教给我的东西。

尽管我全身心地致力于此,结果却是绝对的一无所获。炼金术教程以其复杂难解的语言所断定的事情一件也没有发生过。这种语言乃是些无穷无尽的象征符号,龙的,狮子的,太阳的,月亮的,还有水星的,因为象征性语言有极大的模棱两可性,所以我总有一种误人歧途的感觉。一九七三年,因为毫无进展使我感到绝望,于是我采取了一种极为不当的做法。当时我与马托格罗索州教育厅签订了合同,在该州教授戏剧课程,在题为“绿宝石书板”的戏剧实验课上,我利用了一下我的学生。这件事再加上我曾数次涉足魔幻术的沼泽地,使我在第二年亲身体验到“哪里做便那里结”实乃一句至理格言。此刻我周围的一切彻底崩溃了。

以后的六年,我对与神秘领域有关的一切均持一种相当怀疑的态度。在精神流放期间,我学到了许多重要的东西:我们只接受最初从灵魂深处加以否认的那种真理;我们不应逃避自身的天命;上帝之手虽然严厉,但又是无限慷慨的。

一九八一年,我结识了拉姆教派和我的导师,后者引导我回到了已然为我设计好的道路上。他按他的教授法对我加以训练,与此同时,我重新开始独立研究炼金术。一天晚上,在讲完一堂使人精疲力竭的传心术课程之后,我们交谈起来,我问炼金术士们的语言为什么是那样的空洞和难解。

“炼金术士有三种,”我的导师说,“第一种讲话之所以空洞是因为他们不知道自己正在说些什么;第二种讲话之所以空洞是因为他们知道自己正在说些什么,而且还知道炼金术的语言所指向的是心灵而不是理智。”

“那么第三种呢?”我问。

“他们从未听说过炼金术,但是却通过自身的生活终于发现了哲人石。”

我的导师属于第二种,他决定给我讲授炼金术课程。我发现,这种令我如此气恼和感到茫然的象征符号,乃是进人世界灵魂或是荣格\footnote{荣格(1875-1961):瑞士心理学家,精神病学家。首创分析心理学。}所称的“集体无意识”的惟一方式。我发现了天命和上帝的痕迹,它们是真真切切的,但因其简朴曾被我的理性推论所拒绝接受。我发现,取得元精不是少数人的事,而是地球上所有人类的共同任务。当然,元精并不总是以蛋状物或装着液体的小瓶的形式出现的,但是毫无疑问,我们所有的人都可以深人到世界灵魂之中。

因此,《牧羊少年奇幻之旅》同样是一部象征性作品。通过其篇章,除了转述有关我所学到的一切之外,我还力图向终于理解了宇宙语言的伟大作家表示敬意,例如海明威\footnote{海明威(1899-1961):美国作家,诺贝尔文学奖获得者。代表作有《太阳照常升起》《永别了,武器》讽丧钟为谁而鸣》、《老人与海》等。}、布莱克\footnote{布莱克(1757-1827):英国诗人,版画家。作品主要有《天真的预言》、《天真与经验之歌》和《四天神》等,后期的作品《先知书》陷人神秘主义。}、博尔赫斯\footnote{博尔赫斯(1899-1986):阿根廷诗人,小说家。其重要作品有《交叉小径的花园》《布罗迪埃的报告》等。}(他也将波斯的历史运用于他的一个短篇小说中)、马尔巴·塔罕\footnote{马尔巴·塔罕,系巴西作家儒利奥?塞萨尔?德?梅洛一索萨尔(1895-1794)的笔名。此人致力研究阿拉伯文化,在巴西有相当高的知名度。}等。

为了结束这篇长长的序言和阐明我的导师谈及的第三种炼金术士想要表明的意思,值得回忆一下他在实验室里向我讲过的一个故事。

圣母抱着圣子耶酥决定降临人间并参观一座修道院。所有的神父都深感自豪,他们排成一个长队,逐个来到圣母面前表示敬意。一位神父朗诵了动听的赞美诗,另一位展示了他为《圣经》所绘制的彩画,第三位讲出了所有圣徒的名字。就这样,神父们一个接着一个地向圣母和圣子表示了敬意。

排在队伍最后的是该修道院最贫穷的一位神父,从来没有读过那个时代的充满智慧的著作。他的父母都是普通人,在附近的一个老马戏团里工作,他们教给他的全部东西就是向空中抛球以及其他一些杂耍。

轮到他的时候,其他神父便想结束这场表示敬意的活动,因为这位老杂耍艺人没有任何要事可讲,可能会损害修道院的形象。但是在那位老神父的内心深处,同样也产生了要把自己的某种东西献给耶酥和圣母的强烈渴望。

他有些羞愧,因为他感到了同伴们的责备目光。他从口袋里掏出几个橙子,开始把它们抛向空中,玩起了杂耍,这是他惟一会做的事情。

恰恰是在这个时候,圣子耶酥笑了,并开始在圣母的怀里鼓起掌来。于是圣母将胳膊伸向老神父,让他摸了一下圣子。

\newpage
{\kai {他们走路的时候,耶稣进了一个村庄。有一个女人名叫马大,接他到自己家里。

她有一个妹子名叫马利亚,在耶稣脚前坐着听他的道。

马大伺候的事多,心里忙乱,就进前来说:

“主啊!我的妹子留下我一个人伺候,你不在意吗?请吩咐她来帮助我。”

耶稣回答说:

“马大,马大!你为许多的事思虑烦扰,但是不可少的只有一件,马利亚己经选择那上好的福分,是不能夺去的。”}}

\ \hfill《路加福音》第十章第三十八至四十二节
\chapter*{引子}\addcontentsline{toc}{chapter}{\large\CJKfamily{hei}引子}
炼金家拿起一本商队里某个人带来的书。这本书没有封面,但可以认出它的作者:奥斯卡·王尔德。在翻阅这本书时,炼金家看到了一篇有关水仙花的故事。

炼金家知道这个神话:一个英俊的少年,每天都到一个湖边欣赏自己的美貌。他对自己的容颜如此痴迷,以致于某一天掉迸湖中溺水而身亡。在他落水的地方长出了一株花,人们把它称为水仙花。

然而奥斯卡·王尔德却不是这样结束这个故事的。

他写道,水仙少年死后,山林女神们来到了湖边,发现它由一个淡水湖变成了一个含有咸味泪水的水坛。

“你为什么哭呢?”山林女神们问道。

“我为水仙少年而哭。”湖回答说。

“啊,我们对你为水仙少年而哭并不感到惊讶,”山林女神们说道,“说到底,尽管我们所有女神总在森林里跟在他的后面奔跑,但惟有你有机会能从近处观看他的美貌。”

“水仙少年长得美吗?”湖问道。

“有谁能比你更清楚这一点呢?”山林女神们惊讶地回答道,“他每天都趴在你的边沿欣赏自己的美貌。”

湖静默了片刻,最终说道:

“我是为水仙少年而哭,但我从未注意到他长得很美。

“我为水仙少年而哭,因为每次他趴在我的边沿时,我都能从他的眼睛深处看到映出来的我自己的美丽。”

——“多么美丽的故事。”炼金家说道。

\chapter{第一部}\label{ch1}

那个男孩名叫圣地亚哥。暮色开始降临时,他与他的羊群来到了一座废弃的古老教堂的门前。教堂屋顶很久前就已然塌落,在原先是圣器室的地方,已经长出了一棵巨大的埃及榕。

男孩决定在这里过夜。他把所有的羊赶进残破的大门,随后又用几块木板挡在门上,以防夜里羊儿可能走失。这个地区没有狼,但有一次一只羊在夜里走失了,次日他花了整整一天的时间去寻找这只迷途的羊儿。

他把自己的外衣铺在地上,然后躺了下来,把刚刚读完的一本书当作枕头来用。入睡之前,他提醒自己需要开始读一些更厚的书,读完这样的书要耗费更多的时间,而且夜里当作枕头也会更加舒适。

当他醒来时天还没有亮。抬头向上望去,透过损毁了一半的屋顶,可以看到星星在闪烁。

“真想再多睡一会儿,”他想道。刚刚做了一个梦,与上周所做的完全相同,而且也是在梦结束之前便醒了。

他起身喝了一口酒,然后拿起牧羊拐杖,开始捅醒那些仍在睡觉的羊儿。他己经注意到,只要他一醒,大多数羊也同样开始醒来,仿佛有某种神秘的能量将他的生命与羊儿的生命联在了一起。两年来,他领着羊儿们走遍这块大地,寻找着水和食物。“它们已经太熟悉我了,所以知道我的作息时间。”他喃喃自语道。他思索了片刻,心想也可能恰恰相反,是他熟悉了羊儿们的作息时间。

不过,仍有些羊儿要再过一会儿才醒。男孩一边呼唤着每只羊的名字,一边用牧羊杖一只接一只地捅醒它们。他一直相信这些羊能听懂他的话,所以有时他习惯于把书中的精彩片断读给它们听,或是向它们诉说一个原野上的牧羊人的孤寂与欢乐,还会把在经常路过的一些村庄所见到的新鲜事向它们做一番评论。

但是最近两天,他的话题实际上只有一个:那个女孩。她是一个商人的女儿,住在距此地还有四天路程的那个村庄里。他只是去年到过那里一次。那个商人经营着一间纺织品商店,他总是喜欢让人当着他的面来剪羊毛,以免有人弄虚作假欺骗他。一位朋友介绍他去这家商店,牧羊少年就赶着羊群到了那里。

“我需要卖一些羊毛。”男孩对商人说。

商店里顾客盈门,商人要求牧羊少年等到快傍晚时再说。于是男孩便坐在商店门口碎石铺成的路面上,从搭褪里拿出了一本书。

“我不知道牧羊人还会读书。”他的身边响起了一个女人的声音。

这是一位典型的安达卢西亚地区的少女,一头飘垂着的黑发,眼睛能使人朦胧地联想起昔日的征服者摩尔人。

“那是因为羊群比书本能够教会人们更多的东西。”男孩回答说。两个人接着交谈了两个多小时。女孩说她是商人的女儿,还谈起了村庄日复一日一成不变的生活。牧羊少年则讲述了安达卢西亚的原野,还有他在走过的那些村庄里所见到的种种新鲜事情。不必总要和羊儿们谈话了,这使他感到十分高兴。

“你是怎么学会读书的?”有一刻女孩问道。

“跟所有其他人一样,”男孩回答说,“是在学校里学的。”

“既然你会识字读书,为什么还只是个牧羊人呢?”

男孩找了个随便的借口,回避了她的问题,因为他确信女孩是永远不会理解的。他又继续讲述起一路上发生的种种故事,由于惊奇与害怕,女孩的那双摩尔人似的小眼晴时而睁开时而又闭上。随着时光的流逝,男孩开始盼望那一天永远也不要结束,女孩的父亲一直久久地忙碌,让他在这里等上三天。他察觉到自己正体验着过去从未有过的一种感觉:想在同一个地方长久生活下去的愿望。和这个乌黑头发的女孩在一起,每天的日子将永远不会相同。

但是商人最终还是出现了,吩咐男孩剪下四只羊的羊毛。他付羊毛钱给了男孩,并请男孩明年再来。

现在只差四天就可以重新抵达那个村庄。他感到兴奋,同时也有所不安:也许那个女孩己经把他给忘记了。有许多牧羊人从那里经过和出售他们的羊毛。

“没关系。”男孩对他的羊儿们说道,“我也认识其他地方的别的女孩。”

然而在他的内心深处,他知道是大有关系的。无论是牧羊人还是海员以及流动商贩,他们总会去过一个地方,那里又会有某个人能使他们忘记掉自由自在地周游世界的快乐。

天刚刚破晓,男孩便赶着羊群朝太阳的方向走去。“它们永远不需要做出一个决定,”他想道。“也许这正是它们总要紧紧依随我的原因。”羊儿们感到的惟一需要就是水和食物。 只要男孩知道哪里是安达卢西亚的最好牧场,它们就将永远是他的朋友,哪怕是它们的日子一成不变,哪怕是在日升日落之间耍慢慢度过很长的时间,哪怕是在其短暂的生命中它们从未读过一本书,也不懂人们讲述村庄里发生的种种新鲜事时所使用的语言。有水和食物它们就会高兴,这就足够了。作为回报,它们会慷慨地献出它们的毛,它们的陪伴,还不时地献出它们的肉。

“假如我今天变成了一个魔鬼并把它们一只一只地杀死,它们也只有在几乎全部羊群被杀光之后才会察觉,”男孩想道。“因为它们相信我,而且忘记了相信它们自己的本能。这一切仅仅是因为我能带领它们找到吃的东西。”

男孩开始对自己的这些想法感到惊讶。也许里面长着那棵埃及榕的教堂闹鬼吧。它让他再次做了一个同样的梦,并正使他对一直如此忠实的羊儿心生嫌恶。他喝了一点儿头天晚饭剩下的酒,并把外衣更紧地裹在身上。他知道,再过儿个小时,当太阳升至头顶,炎热将使他不能带领羊群继续在原野上行进。那是所有西班牙人夏季睡觉的时刻,炎热一直要持续到夜晚降临,在整个这段时间里,他不得不总是拎着外衣。不过,每当他想抱怨外衣厚重时,便总会忆起正是因为有了它,清晨他才不会感到寒冷。

“天有不测风云,我们必须总要对此有所准备。”他想道, 并对厚重的外衣心存感谢之情。

外衣的存在有个目的,男孩同样也有自己的目的。两年来他的足迹遍布安达卢西亚平原,已经熟悉这个地区的所有村镇,而旅行便是他生活的大目的。这一次他准备向女孩解释,为什么他这样一个普通的牧羊人会识字读书:他在一所神学院一直读到十六岁。他的父母希望他将来成为一名神父,那会令一个普通的农家引以为荣。父母亲只是为了得到食物和水而辛劳,和他的羊儿们没有什么两样。他学习了拉丁文、西班牙文还有神学。然而,从儿时起,他就梦想了解世界,这远比了解上帝或是了解人类的罪过重要得多。一天下午,他回到家中,鼓足勇气对父亲说,他不想当神父,而是想外出旅行。

“孩子,世界各地的人都到过我们这个村庄,”父亲说道,“他们前来寻找新鲜的事物,可他们继续是原来的老样子。他们一直爬上小山去观看城堡,认为过去要比现在好。他们或是长着金发,或是长着黑皮肤,但和我们村里的人并没有什么两样。”

“可我却没有见过他们那里的城堡。”男孩反驳说。

“这些人在了解了我们的田野和这里的女人之后说,他们喜欢永远在这儿生活。”父亲继续说道。

“我想了解他们生活的地方和那里的女人。”男孩说道,“他们从来没有在我们这里留下来。”

“这些人口袋里装满了钱。”父亲又说,“而在我们这些人当中,只有牧羊人才能去旅行。”

“那我就去当牧羊人。”

父亲没有再说什么。第二天,他给了儿子一个钱袋,里面装着三枚西班牙古金币。

“这是有一天我在田里发现的,本想作为你的财产献给教堂。拿去买你的羊群和周游世界去吧。总有一天你会知道,我们这里的城堡是最有价值的,我们这里的女人是最漂亮的。”

他祝福了儿子。男孩从父亲的眼神中看出,父亲同样也渴望周游世界。数十年来,他为了水和食物操劳,总在同一个地方入睡,使他企图深埋这种渴望,然而尽管如此,这种渴望却依然存在。

朝霞染红了东方,随后太阳便跃出了地平线。男孩回忆起与父亲的那次谈话,感到十分开心。他已经看到过许多城堡,认识了许多女人(但是没有一个与曾等候过他两天的那个女孩一模一样)。他有一件外衣,一本可以用来去换另一本的书,还有一群羊。然而最重要的是,他每天都在实现自己生活的梦想:旅行。一旦厌倦了安达卢西亚的原野,他可以把羊群卖掉,转而成为一名海员。等他厌倦了大海时,一定已经去过了许多城市,认识了许多女人,经历过许多幸福的时刻。

“我不知道如何在神学院里寻找上帝,”他望着冉冉升起的太阳心中想道。只要有可能,他总是会寻找一条新路行进。虽然他从这里路过了不知多少次,却从未到过那座废弃的教堂。世界广阔无恨,假如他让羊群带领自己哪怕走上片刻,他就将会发现更多有趣的事情。“问题是羊儿们不会察觉它们每天都正在走新路,不会注意到草场的变化和季节的更换——因为它们关心的只是水和食物。”

“也许我们所有人也是如此,”牧羊少年思忖道,“连我也是这样,自从认识了那个商人的女儿之后,我就没再想过其他的女人。”他抬头望望天空,估摸午饭之前可以到达塔里法城。在那里,他可以用自己的书换回一本更厚的书,可以灌满一瓶酒,还能刮胡子和理发。他必须为去见那个女孩做好准备,而且不愿去想这样的一种可能性:另外一个牧羊人赶着更多的羊儿,在他之前到了那里向女孩求婚。

“实现梦想让生活变得有意思是完全可能的。”男孩边想边再次望望天空并加快了脚步。他刚刚记起在塔里法住着一位能够释梦的老妇,而那天夜里他做了一个曾经做过一次的梦。

老妇把男孩领进家中最里面的一个房间,一张用彩色塑料条制成的帘子将它与客厅隔开。房间里面有一张桌子、两把椅子和一幅圣心耶酥的画像。

老妇坐下来,并让男孩也坐下。然后她握着男孩的双手开始低声祈祷。

念诵的仿佛是吉卜赛人的祷词。男孩在路上遇到过许多吉卜赛人,他们周游四方,但并不放牧羊群。人们常说,吉卜赛人总靠欺骗他人为生。人们还说,他们与魔鬼订有盟约,还拐骗小孩到他们神秘的帐篷里去充当奴隶。小时候,男孩就因为担心遭到吉卜赛人的拐骗而怕得要死。现在,当老妇握着他的双手时,原有的那种恐惧重又出现了。

“但是这里有圣心耶酥像。”男孩想道,竭力使自己更镇静一些。他不想让自己的双手发抖,不想让老妇察觉出他的恐惧。他暗自默诵了一遍天父经。

“真有意思。”老妇望着男孩的双手说道,然后便又沉默不语。

男孩紧张起来,双手不由自主地开始抖动,老妇察觉到了这一点。男孩赶紧把手抽了回来。

“我不是来看手相的。”男孩说道,已经后悔自己走进了那个家。他想了片刻,觉得最好是交钱走人,哪怕是一无所获。他太看重一个做过两遍的梦了。

“你是来找我解梦的,”老妇说道,“梦是上帝的语言。如果上帝使用的是世俗的语言,我就能够做出解释。但是如果他使用的是心灵的语言,那就只有你自己才能够理解。不管怎样我都要收费。”

又是一个诡计,男孩想道。但是他决定冒险。一个牧羊人总要遇上碰到狼或是干旱的风险,正是这一点才使牧羊人的生涯更富于刺激性。

“我接连两次做了同一个梦,”男孩说道,“梦见我在田野放羊时来了一个小孩,他开始和羊儿们玩耍。我不喜欢有人逗弄我的羊,它们害怕生人。不过,孩子们总能逗弄它们又不使它们受到惊吓。我不知道为什么会这样,不明白羊儿怎么会识别人的年龄大小。”

“回到你的梦上来,”老妇说道,“我的火上还放着锅,况且你只有不多的钱,不能占用我的全部时间。”

“那个小孩继续和羊儿玩耍了一阵儿,”男孩接着说道,显得有些不自在,“突然,他抓起我的双手,把我一直带到了埃及的金字塔。”

男孩停顿了片刻,想看看老妇是否知道埃及的金字塔是什么,但是老妇没有开口。

“在埃及的金字塔,”男孩慢慢地讲出这几个字,以使老妇能够听明白,“那个小孩对我说: ‘如果你来到这里,就会找到一批埋藏着的财宝。’正当他要指给我宝藏的具体位置时,我就醒了。两次都是这样。”

老妇又沉默了一会儿,然后再次拿起男孩的双手仔细地研究了一番。

“现在我不收取你任何费用,”老妇说道,“但是如果你找到了宝藏,我想要其中的十分之一。”

男孩高兴地笑了。因为事关一个宝藏之梦的缘故,他就可以省下他不多的钱来了。老妇准是个吉卜赛人——吉卜赛人都像驴一样地蠢。

“那你就为我解梦吧。”男孩说。

“你先要发誓,发誓将你的财宝的十分之一给我,作为我为你解梦的酬金。”

男孩发了誓。老妇要求他望着圣心耶酥像重复一遍誓言。

“这是一个使用世俗语言的梦,”老妇说道,“我能够解释它,可这个梦非常难解,因此我认为我理应从你的财宝申拿到我的那一份。

“我的解释是这样的:你应该去埃及的金字塔。我从来没有听说过金字塔,但假如一个小孩向你展示过,那就是因为它们确实存在。你会在那里找到那批财宝,它将会使你成为富翁。”

男孩先是感到吃惊,随后便感到气恼。他无需为此而来找这位老妇。但是他终于想到自己可以不用付任何费用。

“我不需要浪费自己的时间听这种话,”他说道。

“所以我才对你说你的梦难解。简单的事物却是最不寻常的,只有智者才能洞悉。既然我不是个智者,所以我必须熟悉其他的技艺,比如看手相。”

“我怎么到埃及去呢?”

“我只是解梦,并不知道如何把梦变成现实。正因为如此,我必须靠我的女儿们养活。”

“如果我到不了埃及呢?”

“那我就拿不到我的酬金了。这将不会是第一次。”

老妇没有再说什么。她要男孩离去,因为她己经为他失去了许多时间。

男孩失望地离去了,并决定再也不相信梦了。他想起还有几件事需要办理。他去商店买了些吃的东西,又把自己的书换成一本更厚的书,然后在广场的一张长凳上坐下来品尝刚买的新酒。这一天天气很热,而酒却因为一种深不可测的玄妙使他的身体略感清凉。羊群被关在城门口他的一位新朋友的羊圈里。他认识这一带的许多人,正因为如此他喜欢旅行。旅行总能结交新的朋友,又不需要日复一日地与他们在一起。当一个人每天看到的总是一些熟面孔时——在神学院就是如此——,这些人最终就会成为你生活中的一部分。因为他们成为了你生活中的一部分,所以他们便开始想改变你的生活。如果你不是像他们期望的那样行事,他们就会厌恶你,因为所有的人对我们应该如何生活都有一个固定的观念。

但他们对自己应该如何生活却从没有任何概念。就像那个给人解梦的老妇,她不知道如何把梦想转变成现实。

他决定等太阳更西沉一些之后,再赶着羊群上路。还有三天时间,他就会与那个商人的女儿见面了。

他开始阅读从塔里法的神父那里换到的新书。书很厚,第一页就讲述了一场葬礼。除此之外,人物的名字非常复杂难记。假如有一天我来写一本书的话,男孩想道,每一次我只介绍一个人物出场,以便让读者无需去记住人物的名字。

当他能集中一些精力进行阅读时——阅读起来很憾意,因为描写的是一场雪地里的葬礼,能带给坐在大太阳底下的他一种清冷的感觉---,一个老翁在他身边坐了下来,开始与他搭汕。

“那些人正在做什么?”老翁指着广场上的人们问道。

“正在工作。”男孩冷淡地回答道,接着便装出一副重新集中精力读书的样子。其实,此刻他正在想象当着那个商人的女儿面剪羊毛的情景,好让女孩目睹他是如何有能力去做些有趣的事情。这一场面他不知想象过多少次了,每一次当他向女孩开始解释羊毛要从后面向前剪的时候,女孩都会听得入神着迷。男孩同样还回忆起几个动听的故事,他边剪羊毛边讲给女孩听。这些故事大部分都是他从书中读来的,但当他讲起来时却要仿佛是自己亲身经历过似的。女孩永远不会知道其中的不同,因为她不识字。

然而老翁却不肯罢休。他说他又累又渴,请求男孩给他喝一口酒。男孩把酒瓶递给他,以为也许老翁会因此而沉默不语。

但是老翁无论如何还是要与他交谈,问男孩正在看什么书。男孩本想对老翁粗鲁行事,换另外一条长凳坐下,但他父亲曾教导他说要尊重比他年长的人,于是他便把书递给了老翁。这样做有两个原因:其一是他不知道书名如何读;其二是如果老翁也不会读,为了不感到羞愧,对方自己就会换另外一条长凳去坐。

“噢……”老人把书接过来左看右看,仿佛它是一件奇怪的物件,然后说道:“这本书很重要,不过十分令人厌烦。”

男孩吃了一惊。老翁也识字,而且己经读过这本书。如果这本书真像老翁说的那样令人厌烦,他还有时间去换另一本。

“这本书讲的东西几乎同所有的书所讲的一样,”老翁接着说道,“认为人没有能力选择自己的命运,还要所有的人都相信世界上最大的谎言。”

“什么是世界上最大的谎言呢?”男孩吃惊地问道。

“在我们人生的某一时刻,我们失去了对人生的控制,人生便转由命运来主宰,这就是世界上最大的谎言。”

“我的情况不同,”男孩说道,“别人想让我当一名神父,而我却决定要当一个牧羊人。”

“这最好不过了,”老翁说道,“因为你喜欢旅行。”

“他猜出了我的想法,”男孩心里想道。与此同时,老翁翻阅着那本厚书,丝毫没有把书还给男孩的意思。男孩注意到老翁穿的衣服很特别,好像是一个阿拉伯人。这种情况在这个地区并不罕见。塔里法距离非洲只有儿个小时的路程,只要乘船就可以渡过窄窄的海峡。阿拉伯人经常出现在这个城市,他们前来购物,每天要做儿次奇怪的祈祷。

“先生是什么地方的人? ”男孩问道。

“是许多地方的人。”

“一个人不可能是许多地方的人。”男孩说道,“我是个牧羊人,到过许多地方,但我只是一个地方的人,一个古城堡附近的村庄,那儿是我的出生之地。”

“那我们可以说我出生在撒冷。”

男孩不知道撒冷在哪里,但又不愿去问,以免因自己的无知而感到羞愧。他又打量了会儿广场,人们来来往往似乎都很忙碌。

“撒冷现在怎么样?”男孩问道,试图寻找出某些线索。

“还是过去的老样子。”

依然没有任何线索。但男孩知道撒冷不在安达卢西亚,否则他早就熟悉了。

“您在撒冷做什么? ”男孩坚持问道。

“我在撒冷做什么? ”老翁第一次开怀大笑起来,“我是撒冷之王呀!”

人常讲出些非常稀奇古怪的事情,男孩想到。有时候最好是和羊儿待在一起,它们沉默无言,只是寻找食物和水。或者最好是以书为伴,书总是在人们想听的时候讲出一些令人难以相信的故事来。但是当和人交谈的时候,他们会讲出某些东西,令你不知道如何继续与之交谈下去。

“我叫麦基洗德。”老翁说道,“你有多少只羊? ”

“足够多的。”男孩答道。老翁非常想了解他的生活情况。

“那我们就面临一个难题了。如果你认为你有足够多的羊,我就无法帮助你了。”

男孩不高兴了。他并没有请求帮助,相反是老翁跟他要酒喝,找话谈,翻阅他的书。

“请把书还给我。”男孩说道,“我要去赶我的羊群上路了。”

“把你十分之一的羊儿给我,”老人说道,“我就教你如何找到宝藏的地方。”

男孩于是又想起了他的梦,突然间一切都变得清清楚楚。老妇没有收取他一文钱,但这个老翁——也许是她的丈夫——,却以提供莫须有的信息为交换条件,想骗取他更多的钱。老翁大概也是个吉卜赛人。

不等男孩开口,老翁便俯下身子,拿起一根木棍,开始在广场的沙地上写了起来。俯身时,他怀里的一个什么东西闪烁了一下,放射出的光芒如此强烈,几乎使男孩什么都看不见了。老翁以一个他这种年纪不该有的极快动作,又用罩袍将那个发光物遮盖住了。男孩的视力恢复了常态,看清了老翁所写的东西。

在这个小城主广场的沙地上,男孩读到了自己父母亲的名字,读到了直至此刻为止他走过的人生历程,童年的玩耍,神学院的寒夜。他还读到了那个商人的女儿的名字,这是男孩过去一直不知道的。有些事情他从未向任何人讲过,比如有一天他曾偷出父亲的枪打死过鹿,还有他的第一次独自的性体验,也竟然都写在了沙地上。

“我是撒冷之王。”老翁说道。

“一个国王为什么要和一个牧羊人交谈呢?”男孩羞愧而又极为敬畏地间道。

“有几个原因。但我要说的是,最重要的是你已具备能力履行你的天命。”

男孩不懂得什么是天命。

“天命就是你一直总希望去做的事情。所有的人在刚步人青年时代时,都知道自己的天命是什么。

“在生命的那一时刻,一切都清清楚楚,一切都是可能的,人们敢于梦想,敢于渴望他们喜欢见到的一切发生在自己的生活之中。然而,随着时光的流逝,一种神秘的力量开始试图证明,实现天命是不可能的。”

男孩并不能完全听懂老翁所说的话,但是他想知道什么是“神秘的力量”,当他再讲给商人的女儿听的时候,她一定会惊讶得目瞪口呆。

“这种力量看似有害,其实它正教导你如何去实现自己的天命。它能锻炼你的精神和砥砺你的意志,因为在这个星球上存在着一种伟大的真理:无论你是谁,或无论你想做什么事,当你真心想得到某种东西时,那是因为这种愿望产生于宇宙的灵魂。这就是你来到世间的使命。”

“哪怕仅仅是四处旅行?或者是与一位纺织品商人的女儿结婚?”


“或是去寻找一批财宝。世界灵魂由人们的幸福或是厄运、羡慕、嫉妒所滋养。实现自己的天命是人们惟一的责任。万物为一。

“当你渴望得到某种东西时,整个宇宙都会协力使你实现自己的愿望。"

两个人都沉默了片刻,望着广场和那里的人们。老翁首先开口问道:

“为什么你要牧羊呢?”

“因为我喜欢旅行。”

广场的一角,有一个卖爆米花的小贩站在自己红色小车的旁边,老翁指着他说道:

“那个卖爆米花的人小时候也一直希望四处旅行,但他却宁肯先买下一辆爆米花车,年复一年地攒钱。等他老了的时候,他会到非洲去上一个月。他永远不会明白,人们总是有条件去实现自己梦想的。”

“他本应该选择当一名牧羊人。”男孩高声说道。

“他想到过这一点,”老翁说,”但是爆米花小贩比牧羊人的地位要高。爆米花小贩有自己的房子,而牧羊人却要露宿野外。作父母的情愿把女儿嫁给爆米花小贩而不是牧羊人。”

男孩心中感到一阵刺痛,他想到了商人的女儿。在她居住的村庄里,应该会有一个爆米花小贩。

“总之,对他们而言,人们对爆米花小贩和牧羊人的看法,竟然变得比自己的天命还要重要。”

老翁翻阅着那本书,漫不经心地读着其中的一页。男孩等待了片刻,然后就以老翁打断他的同样方式也打断了老翁的阅读。

“你为什么要跟我讲这些呢?”

“因为你试图履行自己的天命,而现在差一点就险些要放弃它。”

“你总是在这种时刻出现吗?”

“并不总以这种方式,但我总是会出现的。有些时候,我以一条好的出路、一个好的主意的形式出现。还有些时候,在关键时刻让事情变得更容易些。以此类推吧。不过大多数人意识不到这一点。”

老翁讲到,上个星期他曾不得不以一块石头的模样出现在一个矿工面前。这个矿工放弃了一切去寻找绿宝石,在一条河里一干就是五年,敲碎了九十九万九千九百九十九块石头也没找到一块绿宝石。这时候矿工想要放弃了,其实只差一块石头——仅仅是一块石头——,他就会发现他的绿宝石。因为矿工己经为完成自己的天命牺牲了一切,所以老翁决定帮他一把。老翁把自己变成一块石头,滚动到矿工的脚前。五年来的怒气和失败,使矿工捡起石头便向远处扔去。由于用力很猛,这块石头撞在另一块石头上时就碎裂了,露出了世上最美丽的一块绿宝石。

“世人很早就学习生活的道理,”老人双眼带着某种苦涩说道,”也许正因为如此,他们也很早就放弃了这些道理。世事就是如此。”

男孩此刻想起了他们的谈话是从宝藏开始的。

“流水可以使宝藏露出地面,也可以把它掩埋在地下,”老翁说道,“假如你想知道有关你的财宝的情况,就必须把你的十分之一的羊儿给我。”

“把财宝的十分之一给你不行吗?”

老翁露出失望的神情。

“如果你把还没有到手的东西许诺出去,你就会失去为得到它而努力的愿望。”

男孩于是说出他己经答应将财宝的十分之一给为他解梦的吉卜赛女人。

“吉卜赛人聪明,”老翁叹息道,“无论如何,让你懂得生活中的一切都要有一个代价是件好事。光明斗士试图教给人们的正是这一点。”

老翁把书还给了男孩。

“明天还是这个时候,你把你十分之一的羊儿交给我,我将教你如何去找到那批宝藏。再见。”

接着老翁便在广场的一个角落消失了。

男孩想要读书,但却无法再集中精神。他感到紧张不安,因为他知道老翁说的是实话。

他径直走到爆米花小贩前,买了一包爆米花,同时心里考虑是否应该将老翁所说的告诉小贩。有时候最好是让事情维持现状,男孩想道,于是便没有开口。如果他说出来,Qī.shū.ωǎng.爆米花小贩就会花上三天时间去想是否要舍弃现有的一切,但是他已经十分习惯与他的小车在一起了。

男孩不想让小贩烦恼。他开始毫无目的地在城里闲逛,最后来到了港口。这里有一间小屋,小屋有一扇窗口,有人在那里购买船票。埃及就在非洲。

“要买船票吗?”窗口里面的人问道。

“也许明天吧。”男孩边说边走开了。只要卖掉一只羊,他就可以到达海峡的另一岸。这个念头使他感到一阵惊恐。

“又一个白日做梦的,”售票员见男孩走开了便对他的助手说道,“他没有钱去旅行。”

刚才在售票窗口前,男孩曾想到了他的羊群,现在他害怕回到它们的身边。两年来他己经掌握了牧羊人应该有的全部本领:会剪羊毛,会照顾怀孕的母羊,知道如何保护羊群不受狼的伤害。他熟悉安达卢西亚所有的田野和牧场,清楚他的每一只羊买入和卖出的合理价格。

他决定选一条最绕远的路返回朋友的羊圈。该市也有一座城堡,他决定爬上石头斜坡,到其中的一段城墙上坐下来,从那上面他可以看到非洲。曾有人告诉过他,摩尔人当年就是从这里过来的,占领了几乎整个西班牙许多许多年。男孩憎恶摩尔人,是他们把吉卜赛人带到这里来的。

从他坐着的地方也能俯瞅几乎整个城市,包括他与老翁谈话的那个广场。

“在那个时候遇见他真是讨厌,”男孩想道。他到这里来只是想要找个为他解梦的女人而已,可无论是那个老妇还是那个老翁,都漠视他只是一个牧羊人的这一事实。他们都是孤单老人,己经不再相信生活,不理解牧羊人最后会与他们的羊儿紧密相连。他对他的每只羊都了如指掌,知道哪只羊瘸了腿,哪只羊两个月后要下崽,哪些羊最懒。他还知道如何给它们剪羊毛,如何宰杀它们。假如他决定离开它们,它们将会是很痛苦的。

起风了。他知道这种风,人们把它称作黎凡特风,因为当年成群的异教徒就是与这种风一起来的。在熟悉塔里法市之前,男孩从未想过非洲离这里竟如此之近。这乃是一种巨大的危险:摩尔人有可能重新人侵。

黎凡特风越刮越猛。我要在羊群和财宝之间作出抉择,男孩想道。一个是他已然习惯了的东西,另一个是他想要拥有的东西,他只能选择其中之一。还有那个商人的女儿,但她不像羊群那么重要,因为她并不依赖于他。也许她都不记得他了。男孩确信,如果两天后他没到那里去,女孩也一定不会有所察觉,因为对她来说,每一天都是一模一样的。如果每一天都变得一模一样,那是因为人们不再能感受到美好的东西,但只要太阳穿越天空,人们的生活中就总会出现美好的东西。

“我离开了我的父亲,我的母亲,还有故乡的城堡。他们己经习惯了,而我也同样如此。羊儿也会慢慢习惯我不在它们的身边。”男孩想道。

男孩从上边望了一眼广场。小贩仍然在卖他的爆米花。一对年轻的情侣在刚才他与老翁谈话的长凳上坐下来,久久地亲吻。

“那个小贩……”他自言自语道,却没有把话讲完,因为黎凡特风越刮越猛,他的脸感到了风的强劲。的确,这种风带来了摩尔人,但也同时带来了沙漠和头戴面纱的女人的气味,还带来了为寻求末知物、黄金、历险……以及金字塔而出发的男人们的汗味和梦想。男孩开始羡慕风的自由自在,并意识到他也能够如风一样。除了他自己,没有任何东西能够阻止他。羊群,商人的女儿,安达卢西亚的田野,都只是他迈向天命之路的一个个步伐而已。

第二天中午,男孩带着六只羊和老翁见了面。

“我很惊讶,”男孩说道,“我的朋友立刻买去了我的羊,还说他一生都梦想成为牧羊人。这是个好兆头。”

“事情总是如此,”老翁说道,“我们把这称作为有助原则。如果你是第一次玩牌,几乎百分之百地会赢,这是新手的运气。”

“为什么会这样呢?”

“因为生活希望你去完成自己的天命。”

老翁开始检查这六只羊,发现其中一只瘸了腿。男孩解释说这无关紧要,因为它是最聪明的一只羊,而且能产相当多的毛。

“财宝在哪里呢?”他问道。

“财宝在埃及,靠近金字塔。”

男孩吃了一惊。解梦的老妇也是这样说的,但她没收取任何东西。

“要想找到财宝,你必须随预兆而行。上帝为世上的每个人写出了他应该走的路,你只要读懂他为你所写的就行了。”

就在男孩想要说些什么之前,一只蝴蝶开始在他和老翁之间飞来飞去。男孩想起了他的祖父,在他小的时候,祖父曾对他说过,和螺摔、蠢斯、壁虎和四叶草一样,蝴蝶是好运的征兆。

“完全正确,”老翁说道,他能读出男孩的想法,“正像你祖父教你的那样,这些都是好的征兆。”

老翁随后打开罩在胸部的袍子,男孩被他所看到的东西惊呆了,不禁回想起前一天他所见到的闪光之物。老翁穿着一件纯金制成的胸饰,上面缀满了宝石。

他确实是位国王,所以伪装成这个样子,大概是为了避开盗贼。

“拿着,”老翁边说边把纯金胸饰中心缀着的一白一黑两块宝石取了下来,”它们叫作乌陵和土明。黑的意味着‘是',白的意味着‘否'。当你不能辨别预兆时,它们会帮助你的。记住,你提出的问题永远要客观真实。

“不过在一般情况下,你要尽量自己做出决定。财宝就在金字塔那里,这你已经知道。不过你必须付给我六只羊,因为我帮助你做出了一个决定。”

男孩把两块宝石收进了褡裢。从现在起,他就要自己去做出各种决定了。

“不要忘记万物为一,不要忘记预兆的表现方式,尤其不要忘记追随你的天命一直到底。

“不过在我们分手之前,我想给你讲一个小故事。

“有位商人,把儿子派往世界上最有智慧的人那里,去讨教幸福的秘密。少年在沙漠里走了四十天,终于来到一座位于高山顶上的美丽城堡,那里住着他要寻找的智者。

“我们的主人公走进一间大厅,他并没有遇到一位圣人,相反却目睹了一个热闹非凡的场面:商人们进进出出,每个角落都有人在进行交谈,一个小乐队在演奏轻柔的乐曲,一张桌子上摆满了那个地区最好的美味佳肴。智者正一个个地同所有的人谈话,所以少年必须要等上两个小时才能轮到。

“智者认真地听了少年所讲的来访原因,但说此刻他没有时间向少年讲解幸福的秘密。他建议少年在他的宫殿里转上一圈,两个小时之后再回来找他。

“‘与此同时我要求你办一件事,'智者边说边把一个汤匙递给少年,并在里面滴进了两滴油,。当你走路时,拿好这个汤匙,不要让油酒出来。'

“少年开始沿着宫殿的台阶上上下下,眼晴始终紧盯着汤匙不放。两个小时之后,他回到了智者的面前。

“你看到我餐厅里的波斯壁毯了吗?看到园艺大师花十年心血创造出来的花园了吗?注意到我图书馆里那些美丽的羊皮纸文献了吗?智者问道。

“少年十分尴尬,坦率承认他什么也没有看到。他当时惟一关注的只是智者交付给他的事,即不要让油从汤匙里洒出来。

“‘那你就回去见识一下我这里的种种珍奇之物吧。'智者说道,。如果你不了解一个人的家,你就不能信任他。'

“少年轻松多了,他拿起汤匙重新回到宫殿漫步。这一次他注意到了天花板和墙壁上悬挂的所有艺术品,观赏了花园和四周的山景,看到了花儿的娇嫩和每件艺术品都被精心地摆放在恰如其分的位置上。当他再回到智者面前时,少年仔仔细细地讲述了他所见到的一切。

“‘可是我交给你的两滴油在哪里呢?’智者问道。

“少年朝汤匙望去,发现油已经洒光了。

“那么这就是我要给你的惟一忠告,'智者中最智者说道,‘幸福的秘密在于欣赏世界上所有的奇观异景,同时永远不要忘记汤匙里的两滴油。'”

牧羊少年默不作声。他己经理解了老国王讲的这个故事。一个牧羊人喜欢旅行,但永远不要忘记他的羊群。

老翁望了一眼牧羊少年,双手平放在他的头上做了一些奇怪的手势,然后便带着羊上路了。

塔里法小城的最高处,有一座摩尔人修建的旧城堡,坐在它的城墙上,可以看到一个广场,一个爆米花小贩,还可以看到非洲的一部分。撒冷之王麦基洗德那天下午坐在城堡的墙上,感到黎凡特风吹在了他的脸上。六只羊在他的身边不停地乱动,它们害怕新的主人,如此之多的变化使它们受到了刺激。它们所需要的一切无非只是食物和水罢了。

麦基洗德望着一只正在驶离港口的小船。他再也见不到那个牧羊少年了,如同他收取了亚伯拉罕十分之一的所得之后再也没有见到亚伯拉罕一样。但这就是他的工作。

神是不应该有什么欲望的,因为他们没有天命。然而撒冷之王却衷心祝愿牧羊少年能够取得成功。

遗憾的是他很快就会忘记我的名字,撒冷之王想道,当时我应该多重复一遍我的名字,这样一来,当他谈起我的时候,就会说我是麦基洗德,是撒冷之王。

他望了望天空,多少有些后悔地说道:”主啊,我知道,正如你所说的那样,这是虚荣中的虚荣。不过一位老国王有时需要为自己感到自豪。”

“非洲真是奇怪得很,”男孩想道。

他坐在一个类似酒吧的地方里面,这个酒吧与他在该城狭窄的街道上遇见过的其他酒吧毫无二致。几个男人轮流地抽着一个巨大的烟袋。短短的几个小时之内,他看到了手牵着手的男人,蒙着面纱的女人,还看到阿訇们爬上高高的钟楼,然后开始唱诵,与此同时,周围所有的人全跪了下来,用额头触地。

“异教徒的仪式。男孩自言自语道。小的时候,在自己村庄的教堂里,他总看到一张圣徒圣地亚哥?马塔莫罗斯的画像。圣徒骑在他的白马上,手握一把出鞘之剑,脚下便是和眼前那些人一样的异教徒。男孩感到难受和特别的孤独。异教徒们有着一种邪恶的目光。

除此之外,由于匆忙上路,他忘记了一个细节,只是一个细节,而这一细节却可能在很长的时间里阻碍他找到自己的财宝:这个国家所有的人都只讲阿拉伯语。

酒吧的主人走了过来,男孩指了指另一张桌子的人正在喝的饮料。这是一种苦茶。男孩更喜欢酒。

不过,现在他不应该关心饮料问题,他必须要考虑的只是他的财宝以及如何得到它。卖掉羊群使他口袋里有了相当多的钱,而男孩知道钱乃神奇之物:有了它,无论谁都永远不会孤独。用不了很久,也许只是几天,他就会到达金字塔。一个老翁,戴着纯金胸饰,没有必要为了得到六只羊而骗人。

老翁与他谈过预兆的事。横渡海峡时,他曾想到了预兆。是的,他懂得老翁所说的话,当他在安达卢西亚原野上牧羊时,就已经习惯于通过观察大地和天空来选择要走的道路。他还知道,某种鸟儿的出现意味着附近有一条蛇,某种灌木丛标明距此几公里的地方会有水。这都是羊教会他的。

“如果上帝能把羊群带领得这么好,那他也就能把人带领好。”男孩这样想道,于是心里坦然多了。茶似乎也不那么苦了。

“你是谁?”男孩听到有人用西班牙语问他。

男孩大大地松了一口气。他正在想着预兆,结果就有某个人出现了。

“你怎么会讲西班牙语?”男孩问道。刚来的人是个身穿西式服装的小伙子,但他的肤色表明他应该是本地人,身高和年龄都与男孩不相上下。

“这里几乎所有的人都会讲西班牙语。我们离西班牙只有两个小时的路程。”

“请坐。你要杯饮料,我来付钱。”男孩说道,“同时给我要杯酒,我不喜欢这种茶。”

“这个国家没有酒,”刚来的小伙子说,“此地的宗教禁止喝酒。”

男孩于是告诉对方自己需要去金字塔。他差一点就要讲出财宝的事,但最后还是决定闭口不谈。如果说出来,这个阿拉伯人很可能会索要部分财宝,作为给他带路的报酬。男孩回想起老翁对他讲过的话:不要把还没有到手的东西许诺出去。

“如果你行的话,我想让你带我到那里去。我可以付给你作为向导的报酬。”

“那你想过怎么才能到达那里吗?”

男孩发现酒吧主人正站在附近专心致志地听他们谈话。酒吧主人的出现使他感到不快。不过,他己经找到了一个向导,他不会失去这个机会。

“你必须要穿越整个撒哈拉大沙漠,”刚来的小伙子说道,“要做到这一点我们必须要有钱。我想知道你是不是有足够的钱。”

男孩认为这个问题很是奇怪。但他相信那位老翁,而老翁曾对他说过,当你渴望得到一件东西时,整个宇宙都会协力来帮助你。

男孩从衣袋里取出钱,拿给那个小伙子看。酒吧主人凑了过来也看了看。酒吧主人与小伙子用阿拉伯语交谈了几句,前者显得十分气愤。

“我们走吧,”小伙子对男孩说道,“他不愿意我们继续待在这里。”

男孩松了一口气,站起身来付钱。但是酒吧主人拉住了他,开始不停地讲了起来。男孩身强力壮,不过,他正处身在一个陌生的国度。他的新朋友把酒店主人推到了一边,拉着男孩朝外走去。

“他想要你的钱,”小伙子说道,“丹吉尔不同于非洲其他地方。这儿是一个港口,凡是港口总会有许多小偷。”

男孩信任他的新朋友,是他在关键时刻帮助了自己。男孩掏出钱数了数。

“明天我们就可到达金字塔。”小伙子接过钱来说道,“但是我需要购买两头骆驼。”

两个人开始在丹吉尔狭窄的街道上行走起来。每个街角都有一些卖东西的货摊。他们最后来到了一个大广场的中心,这里正举办集市。成千的人在讨价还价,买进卖出,蔬菜和短剑混杂在一起,地毯旁边是各种样式的烟袋。男孩的目光一直紧紧地盯住他的新朋友,不管怎么说,他所有的钱都在那个人的手里。他本想把钱要回来,但又觉得这样做有失礼貌。他不了解他踏上的这个陌生国度的风俗习惯。

“只要监视着他就行了。”他对自己说道。他知道自己比对方强壮。

在杂乱无章的商品中间,一把他从未见到过的绝顶漂亮的剑突然映入了他的眼帘。剑鞘是银制的,剑柄是黑色的,上面镶嵌着宝石。男孩自己答应自己,等从埃及回来时,一定要买下这把剑。

“请你问问摊主,那把剑要多少钱?”男孩对他的朋友说道。但他察觉到,在他打量那把剑的时候,有两秒钟他因为分神而放松了对此人的监视。

男孩的心变紧了,仿佛胸部突然收缩了一下似的。他不敢扭头向旁边张望,因为他知道将会看到什么。他继续盯着那把漂亮的剑又看了一会儿,直到鼓足了勇气才转过身去。

在他的周围,市场上人来人往,热闹非凡,喧闹声和叫卖声此起彼落,地毯与棒子混放在一起,生菜旁边摆着铜盘,街上的男人们手拉着手,女人们头戴着面纱,异国的食物散发出香味,可是在任何一个地方,确实是任何一个地方,他都没有看到那位朋友的面孔。

男孩还情愿相信他和朋友只是偶然走失了。他决定就留在原地不动,等候对方的归来。没过一会儿,一个人爬上了一座钟楼开始唱诵,所有的人都跪了下来,用额头触地,也随着唱诵。随后,人们像一群勤劳的蚂蚁似的,拆掉摊位四散而去。

太阳开始西沉。男孩久久地望着太阳,直到它隐没在广场周围白色房屋的后面。男孩回想起,当太阳在清晨升起时,他还在另外一个大陆,还是一个牧羊人,拥有六十只羊,并且预定要与一个女孩见面。那天清晨,当他在田野里行走时,他能知道将要发生的一切。

然而现在,当太阳沉落下去的时候,他却身处异国,成了一个陌生国度里的一个陌生人,甚至听不懂人们所讲的话。他已然不是一个牧羊人,已然变得一无所有,甚至没有钱回去和重新开始一切。

“这一切都发生在太阳的一起一落之间,”男孩想道。他为自己感到难过,因为有时候生活中的事情瞬间就会发生变化,人们还来不及加以适应。

他本来羞于流泪,从没有在他的羊儿们面前哭过。不过,现在市场已空无一人,而他又远离了故土。

男孩哭泣起来。他所以哭泣,是因为上帝不公正,竟然以这样的方式回报那些相信自己梦想的人。”当我拥有羊群的时候,我是快乐的,并且把快乐传播给周围的人。人们看到我的时候,总是欢迎我的到来。

“可是现在我却伤心与不幸。我该怎么办呢?我会更加痛苦,我不会再相信任何人,因为有一个人背叛了我。我会仇视那些找到宝藏的人,因为我没能找到自己的那一份。我将永远力图保持住我所拥有的少许之物,因为我过于渺小而无法去拥抱世界。”

男孩打开搭褪,想看看里面自己还有些什么。也许还有一些船上吃剩下来的三明治。但是他只找到了一本厚书、外套和老翁给他的两块宝石。

一看到两块宝石,他便感到了一阵极大的轻松。这两块宝石是他用六只羊换来的,是从纯金胸饰上取下来的。他可以把宝石卖掉,再买一张回程的船票。现在我可要变得更加机灵一点了,男孩边想边从搭褪取出宝石,然后把它们藏进了自己的口袋里面。此地是一个港口,这是那个小伙子告诉他的惟一真话,而一个港口总是到处都有小偷。

现在男孩也明白了酒吧主人为什么感到恼怒:他试图告诉男孩不要相信那个小伙子。”和所有的人一样,我是以我希望的方式,而不是以事情实际发生的方式来看待世界的。”男孩想道。

他望着两块宝石,小心翼翼一块块地抚摸着,感受着它们的温度和平滑的表面。它们是他的财富,只要一触到它们,他就会感到安心了许多。两块宝石使他想起了老翁。

“当你渴望得到某种东西时,整个宇宙都会协力使你实现自己的愿望。”老翁曾这样对他说过。

男孩想要弄明白,这如何能够成为事实。此刻他正在一个空荡荡的集市上,口袋里没有一分钱,这个晚上也没有羊群要他照管。不过,这两块宝石可以证明他曾遇到过一位圣王,这位圣王了解男孩的经历,知道男孩父亲拥有枪支,还知道男孩的第一次性体验。

“宝石分别叫作乌陵和土明,是用来占卜的。”男孩把两块宝石重新放进搭褪,决定做一次实验。老翁说过,所提的问题要明确无误,因为只有知道自己究竟想要什么,宝石才会发挥作用。

于是男孩便问老翁的祝福是否依然伴随着自己。

他取出一块宝石,是意味着“是”的那一块。

“我会找到我的财宝吗?”男孩又问道。

他把手伸进褡裢,正要拿起其中的一块时,两块宝石却都从搭褪的一个破洞处滑落出来。男孩过去从未察觉到他的褡裢有个破洞。他弯身想拣起乌陵和土明,重新把它们放进褡裢里。然而,在看到滑落在地上的两块宝石时,男孩的脑海里又浮现出另一句话来。

“要学会尊重预兆,并服从它的指引。”老圣王曾这样说过。

这是一个预兆。男孩对自己笑了。随后他从地上拣起两块宝石,将它们再次放进搭褪。他不想把破洞缝补上——只要两块宝石愿意,它们可以随时从破洞处滑落出来。男孩已经懂得,为了不逃避自己的天命,有些事情是不应该问的。

“我答应过要自己去做出决定。”他自言自语道。

不过,两块宝石己经告诉了他,老翁仍然在伴随着他,这使他更加有了信心。男孩重新打量了一下空无一人的广场,己经没有了刚才那种绝望的感觉。这里不是一个陌生的世界,而是一个新的世界。

归根结底,他的全部愿望正是要认识那些新的世界。哪怕他永远也到不了金字塔,但与任何一个他所认识的牧羊人相比,他都已经走得更远了。”啊,只要乘两个小时的船,就能发现这么多与过去不同的事物,假如他们能知道这一点该有多好哇。”

展现在他面前的新世界虽然是个空荡荡的市场,但他己经见过它曾充满生机,他永远也不会将其忘记。男孩回想起了那把剑,为了看它一下,他付出了高昂的代价,然而毕竟见到了过去从未见到过的一把如此漂亮的剑。他突然感到,他可以像一个小偷的可怜受害者那样去观察世界,但也可以像一个寻找财宝的冒险者那样去观察世界。

“我是一位寻找财宝的探险家。”在精疲力竭感到睡意袭来之前男孩想道。

一个人把他摇醒了。男孩在集市中心睡了一觉,此刻的广场正准备重新开始它的生机。 他环顾四周,寻找着自己的羊群,接着便察觉到自己已置身于另外一个世界。他没有感到伤心,反倒觉得高兴。他不必再去寻找水和食物,而是可以去寻找财宝。虽然身无分文,但他却对生活充满信任。前一天夜里他已经做出选择,要做一名探险家,和他经常阅读的那些书上的人物完全一样。

他开始不急不忙地在广场上漫步。商贩们架起了他们的货棚,男孩帮助一个甜食商架设了摊位。这位甜食商的脸上露出了与众不同的微笑:他很开心,对生活十分清醒,己做好准备来开始一天的工作。甜食商的笑容使男孩回想起他遇到的那位老翁、那位神奇的圣王的某些事情。”这个甜食商不制作甜食,因为他想外出旅行,或是因为他想和一个商人的女儿结婚。这个甜食商制作甜食,因为他喜欢这个工作。”男孩想道,并发现他也可以做出老翁能做的同样的事情:知道一个人是走近还是远离他的天命。只要观察一下此人就行了。”这很容易,而我过去却从没有察觉到这一点。”男孩想道。

两个人搭好货摊后,甜食商把他制作的第一份甜食递给了男孩。男孩吃得饱饱的,道过谢之后便上路了。当已经离开那里一会儿之后,他突然回想起,架设摊位时,他们两人一个讲的是阿拉伯语,另一个讲的是西班牙语。

但是他们完全能相互理解。

“一定存在着一种超越言词的语言,”男孩想道,“我己经从羊儿那里体验过这一点,现在又从人那里体验到了。”

他正在掌握某些新的东西。有些东西他也曾经历过,但仍然是新的,因为当时虽然经历了却未曾察觉到。他所以未曾察觉到,乃是因为他己习惯了这些东西。”假如我学会理解这种没有言词的语言,我就能理解这个世界。”男孩想道。

“万物为一。”那位老翁曾经说道。

男孩决定不急不忙、不焦不虑地沿丹吉尔狭窄的街道而行,因为只有这样,他才能够察觉到预兆。这需要极大的耐心,但此乃一个牧羊人学会的第一种品德。他又一次发现,他在这个陌生的世界里,恰恰正运用着羊儿教给他的那些知识。

“万物为一。”那位老翁曾经说道。

看到天亮了,水晶店店主感到了每日清晨都要经历的同样烦恼。商店位于一个斜坡的坡顶,很少有顾客由此经过,他在这同一个地方一呆几乎快三十年了。现在要想有任何改变都为时晚矣。他一生学会的惟一事情就是买卖水晶制品。从前曾有过一段时间,很多人都知道他的商店,有阿拉伯商人,有法国和英国的地质学家,还有口袋里总是有钱的德国士兵。在那个时代,出售水晶制品很是赚钱,他曾设想过如何成为一个富翁,如何在年迈时得到漂亮的女人。

后来,好时光一去而不复返,城市也同样如此。休达市的发展超过了丹吉尔市,商机随之发生了转向。邻居们先后搬走了,斜坡地只留下了少数几家商店。谁也不会为逛这儿家商店而爬上斜坡。

但是水晶店店主却别无选择。他从事水晶制品生意三十年了,要改做其他事情己为时太晚。

整个上午,他都望着街道上少数行人的走动。这种情形己有许多年了,他都能知道每位行人由此路过的时间。还差几分钟就要吃午饭的时候,一个外国男孩在他的橱窗前停了下来。男孩的服装同常人一样,可水晶店店主那双经验丰富的眼睛却断定此人没钱。尽管如此,水晶店店主还是决定略等片刻,等男孩离去后再吃午饭。

商店门口的一个招牌说明这里可以使用几种语言。男孩看到一个男人出现在柜台后面。

“如果你愿意的话,我可以把这些杯子擦干净。”男孩说道,“它们现在的这副样子,任何一个顾客也不会想购买。”

男人看着男孩什么话也没讲。

“反过来,你要管我一顿饭。”

男人依然一言不发,男孩感到自己需要做出决定。他的搭褪里有一件外套,在沙漠里他将无需再穿。于是他取出外套,开始用它擦拭那些杯子。半个小时之内,他便把橱窗里的所有杯子擦完。在他擦杯子的这段时间里,进来了两个顾客,买走了几件水晶制品。

全部擦完之后,男孩要求那个男人管他一顿饭。

“我们吃饭去吧。”水晶店店主说道。

他在门上挂起一块休息告示牌子,然后两个人来到斜坡顶上的一个小小酒吧。他们在店里唯一一张桌子旁刚一落座,水晶店店主就微微一笑。

“你什么都不需要擦拭,”他说道,“《可兰经》教义要求必须给挨饿的人饭吃。”

“那你为什么还是让我擦了呢?”男孩问道。

“因为杯子脏了。无论你还是我,都需要清除头脑里的不当想法。”

吃完饭之后,水晶店店主对男孩说道:

“我希望你留在我的店里干活。你擦杯子的时候,进来了两位顾客,这是个好预兆。”

人们总是在讲预兆,牧羊少年想道,但是却不明白他们所讲的究竟是什么。同样,我也不明白这么多年自己是在用一种没有言词的语言同我的羊群说话。

“你愿意为我干活吗?”水晶店店主又问。

“今天余下来的时间可以为你干活。”男孩回答道,“到明天天亮时,我会把店里所有的水晶制品擦干净。作为交换条件,我需要一笔钱,以便明天赶到埃及。”

水晶店老板又笑了起来。

“即使你用整整一年的时间为我擦拭,即使你从卖出的每件商品中都能挣得一笔丰厚的佣金,你也仍然必须借钱才能去埃及。从丹吉尔到金字塔,要穿越几千公里的沙漠。”

万籁无声,城市仿佛己经入睡。集市已散,市场里喧闹的讨价还价声消失了,没有人爬上寺院的尖塔唱诵,也看不到剑柄上镶嵌着宝石的漂亮之剑了。再也没有了希望和冒险,没有了老圣王和天命,没有了财宝和金字塔。仿佛整个世界都沉寂了,因为男孩的心灵沉寂了。没有悲伤,没有痛苦,没有失望;只有透过酒吧小门的茫然目光,只有想要死去的强烈愿望,因为一切都在那一刻永远终结了。

水晶店店主惊恐地望着男孩,似乎那天上午他在男孩身上所见到的快乐突然间便已消失了。

“孩子,我可以给你钱,让你回到自己的家乡去。”水晶店店主说道。

男孩依然沉默不语。后来他站起身,整理了一下衣服,拿起了自己的褡裢。

“我将为你干活。”男孩说道。

久久地再次沉默不语之后,他又说道:

“我需要钱去买一些羊。”


\chapter{第二部}\label{ch2}

男孩为水晶店店主干了近一个月的活,而这并不是一种能使他感到快乐的工作。店主整日在柜台后面哪哪嚷嚷,让他小心水晶制品,不要打碎任何一件。

但男孩继续干了下去,因为店主虽然是个任性的老人,但却并非不公正。每卖出一件商品,男孩都能拿到一笔优厚的佣金,如今他已积攒下一些钱了。这天上午,他算了几笔账,如果天天像现在这样干下去,还需要整整一年他就能够买一些羊了。

“我想给水晶制品做一个展柜,”男孩对店主说道,“可以把它摆放在店的外面,吸引那些从斜坡下面过往的路人。”

“以前我从未做过展柜,”店主回答道,“行人路过时会碰掉水晶制品,它们就会摔碎。”

“以前我在田野上放羊的时候,如果羊碰到一条蛇,就可能会死去。不过,这是羊和牧羊人生活的一个组成部分。”

店主接待了一位想买三个水晶杯子的顾客。现在的生意空前地好,仿佛时光倒流,这条街又成了丹吉尔市最吸引人的地方之一。

“现在的生意已经相当不错了,”在那位顾客离去之后店主对男孩说道,“所赚的钱能使我生活得更好,也能使你很快就可以重新拥有羊群。何必对生活有更多的要求呢?”

“因为我们必须随预兆而行。”男孩似乎是无意地脱口而出,说完之后便立刻感到了后悔,因为水晶店店主从末遇到过一位圣王。

“这叫有助原则,新手的运气,因为生活希望你去完成自己的天命。”老圣王曾这样说过。

然而店主却理解了男孩所说的话。男孩来到他的商店这一简单的事实就是个预兆,随着时间的推移,随着钱柜里的钱越来越多,他对雇用了一个西班牙人一点都不后悔,尽管男孩所挣的钱比其应该得到的要多。当初他一直认为生意不会发生什么变化,所以答应给男孩的佣金甚高,而且他的直觉告诉他,男孩很快就会回去牧羊。

“你为什么要去金字塔呢?”老板问道,目的在于改变展柜的话题。

“因为总有人对我说起它。”男孩这样回答道,以避免提及他做梦的事。如今,那笔财宝已成为他的一个永久性的痛苦回忆,所以男孩避免想起它。

“在我认识的当地人之中,没有一个仅仅是为了去看看金字塔而想要穿越沙漠。”店主说道,“金字塔只是一堆石头罢了,你也可以在自家的后院建造一个。”

“你从未有过外出旅行的梦想。”男孩边说边去接待又一个走进商店的顾客。

两天之后,店主找男孩谈起展柜的事。

“我不喜欢变化。”他说道,”你我都不是富商哈桑那样的人。如果他在一笔生意上出现失误,对他不会有太大的影响。但是我们却要永远为我们的失误去吃苦头。”

的确如此,男孩想道。

“为什么你想要做个展柜呢?”店主问。

“我想尽快地回到我的羊群身边。当好运降临到我们头上时,我们必须要利用它,竭尽全力以它帮助我们的同样方式去帮助它。这叫作有助原则,或是叫作‘新手的运气'。”

店主沉默了一段时间,然后说道:

“先知给了我们《古兰经》,要求我们在生活中只履行五项义务。最重要的一项是,信奉世上只有一个神。其他几项有每天祈祷五次,在斋月要禁食,对穷人要乐善好施。”

店主停了下来。提到先知时,他的双眼擒满了泪水。他是个虞诚的教徒,虽然极缺乏耐心,但仍努力按伊斯兰教义来生活。

“那么第五项义务是什么呢?”男孩问道。

“两天前,你说我从未有过旅行的梦想。”店主回答道,“对所有伊斯兰教徒来说,第五项义务就是去朝圣。在我们一生当中,至少要去一次圣城麦加朝圣。

“圣城麦加比金字塔要远很多。年轻的时候,我宁肯把自己不多的钱积攒下来开这个商店。我想,有一天我成为富翁时,就可以去麦加了。我开始挣钱,但无法放手让任何人来照管这些水晶制品,因为它们是易碎之物。与此同时,我看到许多人从我商店前面经过,朝麦加的方向走去。他们之中有些朝圣者是富人,有仆从和骆驼。但是大部分朝圣者比我还要穷许多。

“所有的人去和回来的时候都兴高采烈,把他们朝圣的象征物置于自家的门上。其中有一位靠替人补靴子为生的鞋匠,他对我说他在沙漠里行走了将近一年的时间,但是当他不得不在丹吉尔穿街走巷购买皮革时,却总感到比前去朝圣更加疲劳。”

“你为什么现在不去麦加呢?”男孩问道。

“因为去麦加的梦想支撑着我活下去,使我能够忍受一成不变的每一天,忍受橱架上这些默默无语的水晶制品,忍受在那间糟糕透顶的餐馆里吃午饭和晚饭。我害怕一旦实现了我的梦想,我就不再会有继续活下去的理由。

“你梦想着羊群和金字塔。你和我不同,因为你希望实现你的梦想,而我只希望拥有去麦加的梦想。我已经儿千遍地想象过穿越沙漠,来到麦加的圣石厂场,在触摸圣石之前应该绕着它走上七圈。我己经想象过哪些人站在我的旁边和我的前面,我们一起进行交谈和祈祷。但是我害怕会大失所望,所以就宁愿只是去梦想。”

这一天,店主同意男孩做一个展柜。

不是所有的人都能以同样的方式看待梦想。

两个月过去了,展柜为水晶店带来了许多顾客。男孩估计,再干上六个月,他就可以回到西班牙并买上六十只羊,然后再买六十只。用不了一年的时间,就能够让他的羊群翻上一番,并能够与阿拉伯人做生意,因为他现在已经会讲那种离奇的语言了。自集市上的那个早上之后,他再也没使用过乌陵和土明,因为埃及对他来说,如同麦加对店主一样,仅仅是一个遥远的梦想而已。与此同时,男孩现在对自己的工作甚为满意,每时每刻都在想着荣归塔里法。

“你要记住,永远都该知道你想要的是什么。”老圣王曾这样说过。男孩知道自己想要的是什么,并且正在为此而工作。也许他的财宝已经来到这一奇异的土地。他先是遇到一个窃贼,后又分文末花使他羊群的数量翻了一番。

男孩为自己感到骄傲。他已经学会了一些重要的东西,比如如何经营水晶生意,懂得了超越言词的语言以及预兆。一天下午,他看到一个人在斜坡顶上,抱怨爬上去之后无法找到一个像样的地方喝点饮料。男孩己经熟悉了预兆的表达方式,便叫来店主对他说:

“我们要卖茶给那些爬上斜坡的人喝。”

“这里有许多人卖茶水。”店主回答道。

“我们可以用水晶杯卖茶水,这样一来,人们喜欢茶,也会想购买水晶杯,因为美最能够吸引人。”

店主看了男孩一会儿,没有作出任何回答。但就在那天下午,做完祈祷和关上店门之后,店主与男孩一起坐在斜坡路上,并邀请男孩与他共吸阿拉伯人使用的那种奇怪的水烟袋。

“什么是你现在所求的东西呢?”水晶店老店主问道。

“我己经告诉过你了。回去以后我要买羊,为此我需要钱。”

老店主又往烟袋里添加了一些新的炭火,然后深深地吸了一口。

“我开这个商店已经三十年了,能识别出水晶杯的好与坏,熟知它们性能的所有细节,了解它们的大小与折光角度。如果用水晶杯盛茶卖,商店就要扩大营业,那样一来,我就不得不改变我的生活方式。”

“这难道不好吗?”

“我己经习惯了我的生活。在你来之前,我认为我在同一个地方丧失了太多的时间,而我的所有朋友都发生了变化,有的破了产,有的发了财。这曾让我感到非常难过。现在我明白了不必如此。商店的大小正好是我一直希望的那种规模。我不希望发生变化,因为我不知道如何变化。我已经非常习惯自己的生活了。”

男孩不知道该说些什么才好。老店主接着说道:

“对我来说,你曾是一种福音。可今天我明白了一件事:没有被接受的所有福音都会变成一种诅咒。我对生活没有更多的需求,而你正迫使我去想象那些过去我从未想过的财富与前景。现在我了解了它们,并且了解了我拥有它们的巨大可能性,结果我的感觉变得比过去更加糟糕,因为我知道我能够拥有这一切,而我却并不想去拥有。”

“幸好当初我没有对爆米花小贩说任何话,”男孩想道。

两个人又吸了一阵儿水烟袋,与此同时太阳渐渐隐没而去。他们是用阿拉伯语交谈的,男孩对自己很满意,因为他会讲阿拉伯语了。曾经有一段时间,男孩曾认为羊儿能够教会他世上的一切。但是羊儿不会教他阿拉伯语。

“世上还应有其他一些事情是羊儿不会教的,”男孩边望着沉默不语的店主边想道。“因为它们只是在寻找水和食物。”

“我认为不是它们在教我,而是我自己学习的。”

“马克图伯。”店主最终说道。

“什么意思?”

“要想弄懂它你就必须要生为阿拉伯人。”店主回答说,”不过差不多可以解释为。注定如此'。”

店主一边熄灭水烟袋的炭火,一边对男孩说,他可以开始用水晶杯卖茶水了。有些时候,阻止生命之河向前流淌是不可能的。

人们在爬斜坡时会感到乏累。于是坡顶上有了一间出售漂亮水晶制品的商店供应清凉的薄荷茶。人们进去饮茶,茶水被装在漂亮的水晶杯里。

“我的妻子从来没有想到过这一点,”有个人说道。随后便购买了儿只水晶杯,因为那天晚上他有客人前来拜访,这些客人定会对这些精美的水晶杯留下深刻的印象。另一个人断言说,用水晶杯喝茶,茶的味道永远会更加可口,因为水晶杯能更好地保留住茶的芳香。第三个人说道,使用水晶杯喝茶是东方的一个传统,因为水晶具有神奇的魔力。

消息很快传播开来,许多人爬上斜坡顶,想见识一下这家商店,因为它在一个如此古老的行业里搞出了一些新的花样。其他一些用水晶杯供应茶水的商店也开张营业,但因为不是地处一个斜坡的坡顶,所以生意总是十分冷清。

不久,店主使不得不又雇用了两个店员。他在进口水晶制品的同时,还进口了大量的茶叶,因为每天都有追求新时尚的男男女女到店里来消费。

六个月就这样过去了。

男孩在日出之前醒了。自从他第一次踏上这块非洲大陆,已经过去了十一个月零九天。

男孩穿上白亚麻布的阿拉伯服装,这是他有意为这一天而买的。接着又包上头巾,用一个骆驼皮制成的圆圈固定住。 最后穿上一双新的凉鞋,悄悄地走下楼梯。

城市还在沉睡。男孩吃了一份自己做的芝麻三明治,用水晶杯喝了热茶,然后便坐在门槛上,一个人吸起水烟袋来。

他默默地吸着,什么也不想,只是静静地听着带来了沙漠气息的风儿那经久不停的响声。吸过烟之后,他把手伸进衣服的一个口袋,凝视了一会儿从里面掏出来的东西。

那是厚厚的一大沓钱,足够买上一百二十只羊,一张回程船票和一张可以使他在他的祖国与他现在所在的国家之间经营商业的许可证。

他耐心地等候着老店主醒来和开门营业。老店主醒来和开门之后,他们两个便一块喝起茶来。

“今天我就要走了。”男孩说道,“我有了买羊的钱,你也有了去麦加的钱。”

老店主没有说话。

“我求你为我祝福,”男孩又说,“你曾经帮助了我。”

老店主继续默默地沏茶。又过了一会儿,他转身面对男孩说道:

“我为你而骄傲。你为我的水晶店带来了生机。但是你知道我不会去麦加,正如你知道你不会回去买羊一样。”

“这是谁告诉你的?”男孩惊讶地问道。

“马克图伯。”老店主简简单单地说道。

然后便为男孩祝福。

男孩回到自己的房间,把所有东西集中在一起,一共装满了三个包。男孩正要离去时,在一个房角他发现了自己牧羊时用的那个旧搭褪。搭褪己被揉磨得不成样子,他几乎早把它忘记了。原来的书和外套还在里面。当他取出外套,想着作为礼物把它送给街上的一个小男孩时,有两块宝石滚落到地上。乌陵和土明。

男孩于是回想起老圣王,并惊讶地察觉到自己己有多么长的时间没有再去想到他了。一年来他不停地干活,一心只想着挣钱,以免垂头丧气地返回西班牙。

“永远不要放弃你的梦想,”老圣王曾经说过,“要随预兆而行。”

男孩从地上拾起乌陵和土明,随即便再次产生了老圣王就近在身边的奇怪感觉。他己辛苦地干了一年的活,预兆表明现在到了他离去的时刻了。

“我回去恰恰还是要成为过去的我,”男孩想道,“可羊儿未曾教过我阿拉伯语。”

但是羊儿却教过他一件更加非常重要的事情:世界上有一种人人都能理解的语言,在让水晶店兴旺起来的整个那段时间里,男孩使用了这种语言。它是一种热情的语言,是以爱和意志所造就的事物的语言,是寻找某种你所希望或是你所相信之物的语言。丹吉尔对他而言己不再是一个陌生的城市,男孩感到,他能够用征服这个地方的同样方式去征服世界。

“当你渴望得到某种东西时,整个宇宙都会协力使你实现自己的愿望。”老圣王曾经说过。

但是老圣王没有说过会有人骗取钱财,没有说过会有无边无际的沙漠,也没有说过会有人知道自己的梦想却不希望去实现。老圣王没有说过金字塔只不过是一堆石头,谁都可以在自家的后院建造一个。而且他还忘记讲了,当你有了钱去买比原来还要多的羊群时,你就应该把它买下来。

男孩拿起褡裢,把它与其他的包放在了一起,然后走下楼梯。老店主正在接待一对外国夫妇,同时还有另外两位顾客在店里转来转去,手持水晶杯喝茶。早晨这个时候能有这样的生意是蛮不错的。男孩从他所在的位置,第一次发现店主的头发很像老圣王的头发。他回忆起那位甜食商的微笑,那是他来到丹吉尔的第一天,当时他正无处可去和无东西可吃。甜食商的微笑也曾使他想到了老圣王。

“仿佛老圣王从那里走过并留下了一个记号似的,”男孩想道。“他们一生中都从未见到过这位圣王。归根结底,老圣王说过,他总是为那些要完成自已天命的人而出现的。”

男孩没有与水晶店店主辞行便出了门。他不想哭着与店主告别,因为这会被别人看到。但是他将会怀念整个这段日子,怀念他在这里所学到的所有美好的东西。他对自己更加充满信心,立志要征服世界。

“可我现在却正要回到我己经熟悉了的田野,再次去牧羊。”他对自己的这一决定不再感到高兴。为实现一个理想我苦干了一年,而这一理想正分分秒秒地失去其重要性。也许是因为它并非是我的梦想。

“谁知道是否像水晶店店主那样更好:永远不去麦加,而是靠要去那里的愿望来支撑生活。”然而男孩手里握着乌陵和土明,这两块宝石给他带来了老圣王的力量与意志。出于一种巧合 或者是一种预兆,男孩想道他来到了踏上非洲大陆第一天去过的那间酒吧。那个骗子己不在了,酒吧主人给他端上了一杯茶。

“我随时都可以重当牧羊人,”男孩想道。“我已经学会了牧羊,永远不会忘记羊儿是什么样子。但是也许我再也没有机会去埃及的金字塔。老撒冷王穿着一件金制的胸饰,知道我过去的一切,是一位真正的圣王,一位智慧的圣王。”

安达卢西亚平原距离他只有两个小时的船程,而金字塔与他之间却横着整个一片沙漠。男孩意识到,也许能以这样的方式来考虑他目前的情况:的确,现在他离他的宝藏已近了两个小时的路程,虽然他为了走完这两小时的路程花费了几乎一年的时间。

“我知道为什么我想回去牧羊,因为我了解羊儿,它们不会给我制造很多麻烦,却能成为我的所爱。我不知道沙漠能否成为我的所爱,但是那里却埋藏着我的财宝。假如找不到我的财宝,我随时都可以回家。但是生活突然给了我足够的钱,而我又有我所需要的全部时间,为什么不去找找看呢?”

男孩此刻感到了一阵无限的快乐。他随时可以再次成为牧羊人,随时可以再次去卖水晶制品。也许世界上还有许多其他埋藏着的财宝,但是他却做过一次内容重复的梦,还遇见过一个圣王。任何其他人都末曾有过这样的经历。

男孩高兴地离开了酒吧。他想起了水晶店店主有一位供应商,此人是通过穿越沙漠的商队把水晶制品运过来的。男孩手里握着乌陵和土明,正是因为这两块宝石,他又重新踏上了寻宝之路。

“我总是在那些想完成自己天命之人的身边。”老撒冷王曾经说过。

他毫不费力地就找到了货栈,想知道金字塔是否真的很远很远。

一个英国人坐在一间散发着牲畜、汗臭和灰尘气味的建筑物里。无法把这样的建筑物称之为货栈,它只不过是个牲口棚罢了。“没料到,在我一生中还不得不来到这么一个地方,”那个英国人边想边心不在焉地翻阅着一本化学杂志,”十年的钻研把我带进了一个牲口棚。”

但是他必须向前走。他一定要相信预兆。他的整个生命和全部研究工作都是为了寻找宇宙使用的惟一语言。最初他对世界语产生了兴趣,后来是宗教,最终则是炼金术。他己经学会了世界语,精通各种宗教,然而却还未成为一名炼金术士。的确他己经揭开了一些重大事物的谜底,但是他的研究遇到了无法再进一步的难关。他曾试图与某位炼金术士建立联系,结果却是枉费心机。炼金术士都是些怪人,他们只想着自己,几乎总是拒绝帮助他人。他们并没有发现被称为哲人石的元精的秘密,因此只好闭口保持沉默,谁知道呢。

他徒劳地寻找哲人石,为此而耗费了父亲留给他的部分遗产。他曾去过世界上那些最好的图书馆,并购买了最重要和最珍稀的有关炼金术的书籍。他在一本书上发现,在许多年前,曾有一位著名的阿拉伯炼金术士访问过欧洲。据说他已有两百多岁了,已经发现了哲人石和长命液。英国人被这个传说吸引住了,但假如不是他的一个从沙漠考古回来的朋友告诉他有位阿拉伯人具有特殊功能的话,那么这个传说就不过是个神话而已。

“此人住在法尤姆绿洲,”他的朋友说道,“据说有两百岁了,能够将任何一种金属变为黄金。”

英国人兴奋得无法自制,立刻取消了他的所有安排,把最重要的书籍收集起来,眼下正坐在这个仿佛是间牲口棚的货栈里。与此同时,外面有一个庞大的商队准备穿越撒哈拉大沙漠,其间要路经法尤姆绿洲。

“我必须要认识那位该诅咒的炼金术士,”英国人想道。这个想法使牲畜散发出的气味似乎变得稍微更能忍受些了。

一个也带着旅行包的年轻的阿拉伯人走了进来,并与英国人打了招呼。

“你要去哪里?”年轻的阿拉伯人问道。

“去沙漠。”英国人回答道,接着便继续看他的书。他不想现在与人交谈。他需要把十年学到的所有东西重温一遍,因为那位炼金术士应该对他进行某种考试。

年轻的阿拉伯人取出一本书来开始阅读。书是用西班牙语写的。“这还不错,”英国人想道。他的西班牙语要比阿拉伯语讲得更好,如果这个男孩也到法尤姆绿洲,在他没有什么要紧的事情可做的时候,就有个聊天的同伴了。

位于埃及法尤姆省东南部,其历史至少可追溯到第十二王朝(前1991-前1786)。在出土物中有许多古埃及文字、希腊文和科普特文的文书残片。

“真有意思,”男孩准备重读那本书开头所描写的葬礼场面时想道,“几乎两年前我就开始读了,可至今也没有把这儿页读完。”尽管现在并没有一个圣王来打断他,男孩也无法做到专心致志。他对自己作出的决定仍有怀疑。不过,他正意识到一个重要的东西:作出决定仅仅是一件事情的开始。当一个人作出一项决定时,实际上他就潜身于一股巨大的洪流之中,这股洪流将把他带往一个在他作出决定时从未想象过的地方。

“当我决定去寻找我的财宝时,从未想象过会在一家水晶店里工作。”男孩想道,用以证实他的推断。同样,加入商队可以是我作出的一个决定,但商队会把我带向何处将一直是一个谜。

在他的面前,有个欧洲人也在读一本书。此人态度冷漠,在他进来时曾轻蔑地看了他一眼。他们本来可以成为好朋友,可那个欧洲人终止了交谈。

男孩合上书,他不想做任何让人觉得他与那个欧洲人相仿的事情。他从口袋里掏出乌陵和土明,开始把玩起来。

那个英国人惊叫道:

“乌陵和土明!”

男孩立刻把两块宝石收进口袋。

“它们是不卖的。”

“这些东西并不很值钱,”英国人说道,“它们只是矿物水晶,仅此而已。地球上有几百万块矿物水晶,不过内行的人都知道它们是乌陵和土明。原先我不知道这个地方还产乌陵和土明。”

“这是一位圣王送给我的礼物。”男孩说道。

英国人没有做声。后来他把手伸进口袋,颤抖着取出了两块一模一样的宝石。

“你提及到了一位圣王。”他说道。

“而你不相信圣王会与牧羊人交谈。”男孩说道,这一次是他想结束这场谈话。

“恰恰相反。当世上其他人拒绝承认一位圣王时,正是牧羊人首先认出了他,所以圣人与牧羊人进行交谈是极有可能的事。”

英国人接着往下说去,不过却担心男孩听不懂他的话: ”《圣经》里有过这种记载。也是这本书教了我有关乌陵与土明的事。它们是上帝允许的唯一占卜之物。祭司们把它们佩戴在一个金制的胸饰上。”

男孩因为自己来到了这间货栈变得十分高兴。

“也许这是一个预兆。”英国人仿佛具有深奥想法之人似的说道。

“谁跟你讲过预兆的事?”男孩的兴趣越来越浓厚。

“生活中的一切都是预兆。”英国人说道,同时合上了他正在阅读的那本杂志。“宇宙存在着一种人人都能懂的语言,但是如今己被遗忘了。除了其他东西之外,我正在寻找的就是这种宇宙语言。我正是因此才来到这里,因为我必须要找到一个懂得宇宙语言的人,他是一位炼金术士。”

他们的谈话被货栈老板所打断。

“你们两个真有运气,”胖胖的阿拉伯人说道,“今天下午就有一个商队出发去法尤姆。”

“可我是要到埃及去。”男孩说道。

“法尤姆就在埃及。”货栈老板说道,“你算什么阿拉伯人?”

男孩说他是西班牙人。英国人甚感满意:男孩虽然穿着阿拉伯人的服装,但至少是个欧洲人。

“他把预兆称作‘运气'。”胖胖的阿拉伯人走后英国人说道,“假如可能的话,我要撰写一部巨型的百科全书,谈谈‘运气'和‘巧合'这两个词。宇宙语言就是由这两个词写成的。”

他对男孩说,他能与手里握着乌陵和土明的人相遇绝非是”巧合”。他问男孩是否也要去寻找那位炼金术士。

“我是来寻找财宝的。”男孩说完便马上感到了后悔。不过英国人仿佛并末看重这件事。

“从某种角度上看,我也是如此。”英国人说道。

“我根本不知道什么是炼金术。”当货栈老板开始唤他们出去时男孩说道。

“我是这支商队的领队。”一个蓄着长须长着一双黑眼睛的男人说道,”我掌握着商队每个人的生死大权,因为沙漠就像一个任性的女人,有时候会让男人发疯。”

商队差不多有两百个人,还有两倍于人的牲畜——骆驼、马、驴、家禽。英国人有几只箱子,里面装满了书。人群中有妇女和儿童,还有一些腰里挎剑、肩上扛着长枪的男人。嘈杂声响成一片,领队不得不把他的话重复了好几遍,以便让所有的人都能听明白。

“我们当中有各种各样的人,他们信奉着各不相同的神。我唯一信奉的神是真主安拉,我向他发誓,我将尽一切所能,把事情做到最好,以便再一次征服沙漠。我现在希望,你们每个人也向自己心底里所信奉的神发誓,无论在任何情况下都要服从我。在沙漠里,不服从就意味着死亡。”

人群里发出一阵低语声,每个人都小声地向自己所信奉的神发誓。男孩向基督发了誓。英国人保持了沉默。低语声持续的时间比简单地发一次誓言要长些,因为人们也在祈求上天的保佑。

一声长长的号音响了起来,众人纷纷骑上各自的牲畜。男孩和英国人买了骆驼,他们爬上驼背时都有些吃力。男孩为英国人骑的骆驼感到难过,因为它要驮着几箱沉重的书籍。

“根本不存在什么巧合。”英国人说道,企图继续他们在货栈里开始的那场谈话。“我所以到这里来,是因为我的一个朋友知道有一个阿拉伯人……”

但是商队已开始行进了,男孩不可能听到英国人正在说些什么。不过,男孩很清楚英国人想要说些什么:把一件事与另一件事联系在一起的神秘链条,正是这根链条使他成了一个牧羊人,让他两次做了同一个梦,令他来到了靠近非洲的一个城市,并且在广场上遇到了一位圣王,然后是他的钱被骗走,结果让他认识了一个水晶店店主,接着又……

“一个人越是接近梦想,天命就越加变成他生存的真正原因。”男孩想道。

商队开始向西方前进。他们清晨出发,太阳最毒的时候停下来,傍晚时分再重新上路。英国人大部分时间都消磨在读书上,男孩很少与他交谈。

于是男孩便静悄悄地观察牲畜和人在沙漠里行进的情景,现在的一切都与出发时的那一天大不相同了:当时人群混乱,吵成一团,孩子们的哭闹声、牲畜的嘶鸣声与向导和商人们紧张的吆喝和命令声交织在了一起。

然而在沙漠里,有的只是无休无止的风声和牲畜的脚蹄声,余下的便是一片沉寂,连向导之间也很少说话。

“我已经穿越过许多次这片沙漠了,”有天晚上,一个赶驼人说道,“但是这片沙漠太大了,地平线太远了,让人感到自己十分渺小和不想开口讲话。”

虽然过去从未踏上过沙漠,男孩依然能明白赶驼人话中的含意。每当观看大海或是烈焰时,他便能几个小时保持沉默,脑子里什么都不想,完全沉浸在它们的广阔无根和无比威力之中。

“我从羊儿那里学到了东西,我从水晶那里学到了东西,” 男孩想道,“我同样也可以从沙漠那里学到东西。我认为沙漠更加古老与智慧。”

风一直刮个不停。男孩回想起他坐在塔里法一个城堡上感受到这种风的那一天。假如他不来非洲,也许现在他正轻轻地抚弄着他的羊儿们的皮毛,羊儿们还在安达卢西亚的原野上寻找食物和水。

“它们如今已不再是我的羊了。”男孩并无怀恋地自言自语道,“它们大概己经习惯了新的主人,把我忘记了。这很好。谁像羊儿那样习惯了旅行,谁就知道总有一天需要出发。”

随后男孩又想起了那个商人的女儿,并且断定她已然结婚。说不定是与一个爆米花小贩,或者是一个也会读书识字和会讲出奇故事的牧羊人,毕竟他不该是惟一能够具有这种本事的牧羊人。男孩被自己的这种预感所鼓舞:也许他也正在学习宇宙语言,这种语言能知道所有人的过去与未来。”预感”,这是母亲的习惯说法。男孩开始理解,预感就是灵魂瞬间潜入宇宙的生命洪流之中,世上一切人的经历都在这一洪流中相互联系在一起,而我们便能通晓一切,因为一切都己注定如此。

“马克图伯。”男孩说道,并回想起了水晶店店主。

沙漠有时是沙土,有时则是石头。如果有一块石头挡路,商队便绕过它继续前行。如果遇到一大片岩石,商队就要绕上一个大圈子。如果对骆驼的蹄子来说沙子太细,商队就会找沙子比较坚硬的地方通过。有时候,在原来应该是个湖泊的地区路面满是盐粒,牲畜便不肯前行,赶驼人就要跳下驼背,卸下货物,然后背在自己身上,通过这个危险的地段,接着再把货物重新装载在牲畜身上。如果一个向导病了或是死了,赶驼人就通过抓阎选出一个新的向导。

但是这一切均缘于一个唯一的理由:无论要绕行多少弯路,商队永远朝着同一个方向前进。克服了障碍之后,商队便会重新面向那颗指明绿洲位置的星斗。清晨人们看到那颗星斗在天空中闪烁时,就知道它指向的是一个有女人、水、椰枣树和棕悯树的地方。只有英国人察觉不到这一切,他大部分时间都在埋头阅读他的书籍。

男孩也有一本书,旅程开始的头几天他也曾试图阅读过。但他发现,观察商队和倾听风声要比读书有意思多了,所以当他更好地了解了自己的骆驼并且喜欢上它时,就把书扔掉了。书乃是个毫无必要的负担,虽然男孩曾迷信地认为开卷有益。

男孩与总走在他身边的赶驼人终于成了朋友。入夜之后,当他们围坐在簧火旁时,男孩常把自己当牧羊人时的种种奇遇讲给赶驼人听。

在一次这样的聊天中,赶驼人也讲述了自己的生活经历。

“原先我住在开罗附近的一个地方,”他说道,“我有我的菜园和我的几个孩子,过着一种直到我死去的那一天也不会发生变化的生活。有一年收成比往常都好,我们全家人全去了麦加。我履行了我一生惟一尚需完成的义务,可以安宁地死去了,而我喜欢这样。

有一天,大地开始震动,尼罗河水漫出了堤岸。本来我以为只会发生在别人身上的事情最终轮到了我的头上。我的邻居们害怕洪水会使他们失去自己的檄揽树,我的妻子担心我们的孩子会被洪水冲走,而我生恐看到我所拥有的一切都毁于一旦。

“但是毫无办法。土地无法耕种了,我不得不另谋生路。如今我成了赶驼人。不过由此我明白了安拉的教诲:任何人都不必担心未知的事情,因为谁都有能力获得他所渴望和所需要的一切。

“我们只是担心失去我们已经拥有的东西,无论是我们的生命还是我们的土地。但是,当我们领悟到我们生命的进程和世界历史的进程都是由同一只手所写定时,这种担心就会消失。”

有些时候,不同的商队会在夜间相遇。仿佛一切真是仅由一只手写定似的,其中的一支商队总是有另一支商队正需要的东西。赶驼人相互交换有关风暴的信息,并围坐在簧火边讲述沙漠的种种故事。

还有些时候,会出现头戴风帽的神秘男人,他们是为商队侦探行进路线的贝都因人,负责向商队提供有关强盗和蛮族部落的信息。这些人身着黑衣,只露出一双眼睛,无声无息地赶来,又无声无息地离去。

就在这样的一个夜晚,那位赶驼人来到男孩和英国人围坐的簧火旁。

“有传言说,部落之间打起仗来了。”

三个人都陷人了沉默。男孩发现,尽管谁都没有讲任何一句话,却能感到一种恐惧的气氛。他又一次体验到了那种没有言词的语言,即宇宙语言。

过了一段时间,英国人问是否存在着危险。

“进入了沙漠的人,就不可能回头了。”赶驼人说道,“既然不能回头,我们就只应该关心最好的前进方法。其余的事情就交由安拉管了,包括危险。”

他以”马克图伯”这个神秘的字眼结束了他的话。

“你需要多注意观察商队。”赶驼人走后男孩对英国人说道,“商队绕了许多圈子,却始终把同一个地方作为方向。”

“你应该多读些了解世界的图书,”英国人回答道,“图书与商队是一模一样的。”

由人和牲畜组成的庞大商队开始加速前进。过去人们习惯于在晚间围着簧火聊天,现在不仅白天都沉默不语,连晚间也都开始闭口无言了。有一天领队决定,为了不引起外界对商队的注意,夜间不再点燃簧火。

为了抵御夜间的寒冷,人们用牲畜围成了一个圆圈,所有的人都睡在圆圈中央。领队还在商队周围安排了武装哨兵。

一天夜里,英国人睡不着觉。他叫醒了男孩,两个人开始沿着宿营地四周的沙丘散步。一轮圆月当空,男孩向英国人讲述了自己的经历。

英国人被水晶店的故事深深吸引,自从男孩开始在那里工作之后水晶店的生意便蒸蒸日上。

“这便是推动一切事物的基点,”英国人说道,“在炼金术中被称作世界灵魂。当你全心全意期望得到某种东西时,你就更加接近了世界灵魂。它总是一种积极向上的力量。”

他还说,这不只是人类独有的一种天赋,世上万物也都具有灵魂,无论是矿物、植物、动物,甚至一个简单的念头,都莫不如此。

“地球上所有的事物都一直在变化,因为地球是有生命的,也具有灵魂。我们是这一灵魂的组成部分,可却极少知道它总是在帮我们的忙。不过,你应该明白,在水晶店里,甚至连那些水晶杯都在为你的成功而协力相助。”

男孩望着圆月和白沙,半晌没有讲话。

“我一直看着商队在沙漠中行进,”男孩终于开口道,“发现商队与沙漠讲着同一种语言,因此沙漠才允许商队通过自己。沙漠将检验商队迈出的每一步,以便观察它是否与自己完全协调一致。如果商队做到了与沙漠完全协调一致,它就必将到达绿洲。

“假如我们之中的一个人以十足的勇气来到这里,但却不懂得这种语言,那么在第一天他就会死去。”

两个人一起继续望着圆月。

“这就是预兆的神奇之处。”男孩接着说道,“我目睹了向导们如何看懂沙漠的预兆,商队的灵魂如何与沙漠的灵魂对话。”

过了一段时间,英国人开口了。

“我需要多注意观察商队。”他终于说道。

“我需要读读你的书。男孩说道。

这些书都很古怪,讲的全是汞、盐、龙和国王的事,男孩一点也看不懂。不过,其中有似乎所有的书都在重复着的一种观念:万物都是唯一一物的表现形态。

在其中的一本书里,男孩发现炼金术最重要的字句只有短短的儿行,书写在一块简单的绿宝石上。

“这就是绿宝石书板。”英国人说道,并为能教给男孩一些东西而感到骄傲。 

“既然如此,这么多书又都是干什么用的呢?”男孩问道。

“为了帮助理解这儿行字句。”英国人回答说,但并不非常相信自己所说的话。

最使男孩感兴趣的是讲述著名炼金术士故事的那本书。这些人整整一生都在实验室里致力于金属的提纯。他们相信,如果一块金属经过许多年、许多年的烧炼,奇-书-网最终就会摆脱其固有的全部特性,余下来的则仅仅是世界灵魂。这种唯一之物使得炼金术士们可以理解地球上的任何事物,因为它乃是令万物相通的一种语言。炼金术士们将这一发现称作元精,由部分液体和部分固体所组成。

“想要发现这种语言,观察人和预兆难道还不足够吗? ”男孩问道。

“你有一种把一切都简单化的癖好。”英国人气恼地回答道,“炼金术是一门严肃的学问,每一个步骤都必须完全按照导师们所教的程序进行。”

男孩发现,元精的液体部分叫作长命液,除了可使炼金术士们长生不老之外,还能医治百病。固体部分则称为哲人石。

“发现哲人石可不容易。”英国人说道,“炼金术士们守在实验室里许多年,观察烧炼金属的火焰。由于他们如此专注地进行观察,世上所有的虚荣就渐渐地被弃之脑后了。他们会在一个美好的日子里发现,提炼金属的最终结果是使他们自身得到了净化。”

男孩回想起水晶店店主。店主曾经说过,擦拭水晶杯是件好事,让他们双方都摆脱了不当想法。男孩越来越相信,炼金术可以在日常的生活中学习到。

“此外,”英国人说道,“哲人石具有一种迷人的特性:一小片哲人石就能把大量的金属点化成黄金。”

从这句话开始,男孩对炼金术产生了浓厚的兴趣。他想,只要有一点耐心,他就能够把一切都变成黄金。他读过几位点金成功人士的传记,比如埃尔维西奥、埃利亚斯、富尔卡内利、热伯尔。这些人的故事很是迷人:他们都终于完成了自己的天命。他们四处旅行,遇到了智者,在怀疑者面前展示奇迹,拥有哲人石和长命液。

但是当他想要学会获取元精的方法时,却完全糊涂了,因为书中只有一些图案、用密码写成的说明和晦涩难懂的文字。

“为什么他们讲得这么难懂?”一天晚上男孩问英国人。他发现英国人同样也有些不耐烦,并且想把他的书要回去。

“为的是只让那些有责任读懂它的人才能读懂。”英国人说道,“你想想看,假如所有的人都能把铅变成金子,那么金子很快就不值一文了。

“只有那些坚持不懈的人,只有那些深人钻研的人,他们才能得到元精。正因为如此,我才来到这沙漠的中央,为的是找到一位真正的炼金术士,他会帮助我破解那些密码。”

“这些书是什么时候写成的?”男孩问道。

“许多世纪之前。”

“那个时代还没有发明印刷,”男孩争辩说,“没有办法让所有的人都了解炼金术。为什么这种语言如此古怪,并且配上插图呢?”

英国人什么都没有回答。他说几天来他一直在注意观察商队,却没有发现任何新的东西。他唯一注意到的事情就是有关打仗的议论越来越多。

在一个美好的日子里,男孩把书还给了英国人。

“你学到了很多东西吧?”英国人满怀期望地问道。他需要能与某个人聊聊天,以便忘却对打仗的恐惧。

“我学到了世界有一个灵魂,谁理解了这个灵魂,谁就能理解万物的语言。我学到了有许多炼金术士完成了他们的天命,最终发现了世界灵魂,哲人石,长命液。

“我尤其学到了这些事情都极其简单,以致可以书写在一块绿宝石上面。”

英国人大失所望。他的多年研究,那些神奇的符号,那些晦涩难懂的文字,还有实验室里的种种设备,这一一切都没给男孩留下深刻的印象。他的灵魂大概过于原始,无法理解这些东西,英国人想道。

英国人取回自己的书,将其放进骆驼载着的旅行包里。

“你再去观察你的商队吧,”英国人说道,“它也没教会我任何东西。”

男孩重新注视着沉寂的沙漠和牲畜扬起的沙尘。”每个人都有自己的学习方法,”男孩自言自语道,”他的方法不同于我的方法,而我的方法也不同于他的方法。但是我们俩都在追寻自己的天命,因此我尊重他。”

商队开始日夜兼程。蒙面的贝都因报信人不断出现,已成为男孩朋友的那位赶驼人解释说,这是因为部族之间己经开始打起仗来了。商队要运气极好才能到达绿洲。

牲畜们现已疲惫不堪,人们也变得越来越沉默。骆驼的简单一声嘶叫——在过去不过就是一声嘶叫而已——,如今都令所有的人胆战心惊,<网罗电子书>因为这可能是有人要来进行侵袭的信号。在这种时候,沉默在夜间便更令人感到恐怖。

然而,那位赶驼人对打仗的威胁似乎并不十分在意。

“我现在还活着,”在一个既没有簧火也没有月光的夜晚,赶驼人边吃着一盘蜜枣边对男孩说道,”当我吃东西的时候,我就一心一意地吃。走路的时候,我就只管走路。如果我必须要打仗,那么这一天和其他任何一天一样,都是我死去的好日子。

“因为我既不生活在过去里,也不生活在将来之中,我所有的仅仅是现在,我只对现在感兴趣。假如你能总是把握着现在,那你就会成为一个幸福的人。你将会发觉,沙漠中存在着生命,夜空里有着星星,战士们所以打仗因为这是人类生活的一个组成部分。生活是一个节日,是一场盛大的聚会,因为它永远是又仅仅是我们正在度过的现在时光。”

两天后的一个夜晚,正要准备睡觉时,男孩朝一颗星星的方向望去(夜间商队一直追随着这颗星星行进)。他发现,地平线比往日更低了一些,因为沙漠上空有着数百颗星星。

“那里就是绿洲。”赶驼人说道。

“那为什么我们不马上赶到那里去呢?”

“因为我们现在需要睡觉。”

男孩睁开双眼时,太阳已开始出现在地平线上。在他面前,夜间闪烁着众多小星星的地方,延伸着一排没有尽头的椰枣树,覆盖了沙漠的整个正前方。

“我们终于到了!”英国人欢呼道,他也是刚刚醒来的。

然而男孩却没有做声。他己学会了沙漠的沉默,只是满意地望着眼前的椰枣树。他必须还要走许多的路才能到达金字塔,在将来的某一天,这个清晨不过只能成为一种回忆罢了。但是而今,它正是现在的时刻,是赶驼人曾经说过的节日,所以男孩要连同过去的教训和未来的幻想,努力把握好它。将来的某一天,成千上万棵椰枣树的景象也只会成为一种回忆。然而对他而言,在目前的这一时刻,它却意味着荫凉、水和躲避战火的庇护所。如同骆驼的一声嘶叫可能变化为危险,那一排椰枣树可能意味着一个奇迹。

“世界讲着许多种语言,”男孩想道。

“时光飞逝,商队也同样如此。”看到数以百计的人和牲畜抵达绿洲时,炼金术士心中想道。绿洲的人们跟在刚刚到达的人群后面不停地喊叫,扬尘遮蔽了沙漠的太阳。见到了陌生人的孩童们兴奋地蹦跳起来。炼金术士看到部落的首领们走近商队的领队,彼此进行着长时间的交谈。

但是炼金术士对这一切都不感兴趣。他己经见过许多人的到来和离去,而绿洲和沙漠却依然如故。他见过国王和乞丐踏上这些因为风的缘故总在改变模样的沙地,而沙子却依旧是他小时候所熟悉的沙子。尽管如此,看到所有旅行者在满目黄沙和蓝天之后,眼前出现了椰枣树的绿色时所体验到的喜悦,炼金术士的内心深处依然无法抑制地分享了一点他们的欢乐。“也许上帝创造沙漠的目的就在于使人能够面对椰枣树而微笑。”炼金术士想道。

随后,他便决定将注意力集中在更为实际的事情上来。他知道,他应该将他的部分秘密传授给这支商队里的某一个人,预兆己经告诉了他这一点。现在他还不清楚此人是谁,但是只要一见到此人,他的那双富有经验的眼睛就能将其辨认出来。他期望这个人也具有他前一个弟子的那种能力。

“我不知道为什么这些事情必须要用口对着耳朵去传授,”炼金术士想道。这恰恰不是因为这些事情乃是秘密,上帝总是慷慨地向所有的人揭示出他的秘密。

炼金术士对此只有一种解释:这些事情所以必须要这样传授,乃因为它们是由纯粹生命所造成,而这种类型的生命难以用图画与文字所捕捉。

由于人们执迷于图画与文字,最后便忘记了宇宙语言。

新来的人立刻被带到了绿洲部落首领们的面前。男孩无法相信他所看到的景象:绿洲并非几棵棕榈树围着一眼水井——就像有一次他在一本故事书里读到的那样,而是比西班牙的几个村镇还要大出许多。它拥有三百眼水井,五万棵椰枣树,其间散布着许多五颜六色的帐篷。

“仿佛是到了《一千零一夜》里。”急于马上找到炼金术士的英国人说道。

一群孩子立刻把商队围了起来,好奇地打量着牲畜、骆驼和新到的人们。男人们想知道商队是否见过有人打仗,妇女们彼此之间争购着商人们带来的布匹和宝石。沙漠的沉寂似乎已成为一个遥远的梦境,人们不停地说着、笑着、喊着,仿佛走出了一个灵性世界,重又回到了人世之间,甚感轻松和快乐。

而前一天,商队还保持着戒备。赶驼人对男孩解释说,沙漠中的绿洲一向被视为中立地区,因为那里大部分居民是妇女和儿童。沙漠两侧都有绿洲,战事只在沙漠中进行,使绿洲成为避难之所。

商队领队费了一些力气才把商队的所有人召集起来,然后便开始发布指示。商队将在绿洲停留,直至部落之间的战争结束为止。作为客人,他们将与绿洲的居民共用帐篷,绿洲的居民将把最好的位置让给他们,这出于主人的好客规则。然后他要求所有的人,包括他自己的卫兵,把武器交给部落首领所指定的人。

“这是战争的规则。”商队领队解释说,“通过这种方式,绿洲就能不成为军队或战士的庇护所。”

令男孩大吃一惊的是,英国人从口袋里取出了一把镀铬的手枪,交给了前来收缴武器的人。

“你带一支左轮手枪干什么?”他问道。

“为了使我不担心信任他人。”英国人回答说,“我很高兴自己抵达了最后的目的地。”

男孩想到了自己的财宝。他越是接近自己的梦想,事情就变得越加困难,老圣王称作是”新手的运气”的那种东西就不再起作用了。男孩知道,现在重要的是追寻自己天命之人的毅力和勇气要经受住考验。因此他不能匆匆忙忙,不能失去耐心,如果这样的话,最终他就会看不到上帝在他的路上设置的预兆。

“上帝在我的路上设置的预兆,”男孩对自己的这一想法感到惊讶。在此之前,他一直把预兆看成是世间的一件事情,就像吃饭和睡觉,就像寻找爱情或是得到一个工作。他从未想过,这乃是上帝为了指出他应该去做什么而使用的一种语言。

“不要失去耐心,”男孩对自己重复道,“正如赶驼人所说的那样,该吃饭时吃饭,该走路时走路。”

第一天,由于疲劳,所有的人都人睡了,包括那位英国人。 男孩和另外五个年龄几乎与他一般大的小伙子睡在一个帐篷里,离英国人住的地方很远。这五个小伙子都来自沙漠,很想知道大城市里的故事。

男孩讲述了他当牧羊人时的往事。正当他要谈起他在水晶店的经历时,那位英国人走进了他们的帐篷。

“我找了你整整一个上午,”英国人边说边把男孩拉出了帐篷,“我需要你帮助我找到炼金术士的住处。”

两个人最初试图自己去找。一位炼金术士的住处应该与绿洲其他人的不同,在他的帐篷里,很可能有一座始终燃烧着的炉子。他们走了相当多的地方,直至确信绿洲要比他们所能想象的还要大出许多许多,有着数百座帐篷。

“我们几乎浪费了整整一天的时间。”英国人与男孩一起在一口井边坐下时说道。

“也许我们最好是打听一下。”男孩说道。

英国人不想让别人知道他到绿洲来的目的,所以相当地犹豫不决。但最后他还是同意了,并请男孩前去打听,因为男孩的阿拉伯语讲得比他好。一位妇女刚好带着羊皮囊到井边取水Qī.shū.ωǎng.,男孩便朝她走了过去。

“下午好,夫人。您能告诉我,绿洲这里的一位炼金术士住在什么地方吗扩男孩问道。

那位妇女说她从未听人讲起过炼金术士,接着便立刻离去了。不过在离去之前,她告诫男孩不该和身着黑色服装的女人搭话,因为她们都已结了婚。他必须要恪守传统。

英国人极其失望。他的全部旅行毫无收获。男孩也感到难过,因为他的同伴也在追寻自己的天命。老圣王曾经说过,当一个人追寻自己的天命时,整个宇宙都会努力使他得到他

所希望的东西。老圣王是不会错的。

“以前我从未听人讲起过炼金术士,”男孩说道,“不然的话,我会尽力帮助你的。”

英国人的眼睛闪烁一亮。

“这就对了!也许这里没人知道什么是炼金术!你去打听一下,这里有没有一位能医治百病的人!”

又有几位身着黑色服装的妇女来井边汲水,无论英国人怎样坚持,男孩也并未与她们搭话。直到一个男人走过来时,男孩才上前询问道:

“你认识这里的什么人会治病吗?”

“安拉会医治百病。”那个男人说道,显然他对外国人感到惊慌,“你们是在找巫师。”

念诵几句《古兰经》的经文之后,他便继续走他的路了。

又有一个男人走了过来。此人年纪较大,只带着一只小水桶。男孩向他重复了一遍刚才的问题。

“你们为什么要认识这种人呢?”那个阿拉伯人以问代答。

“因为为了见他,我的这位朋友走了好几个月。”男孩回答说。

“如果绿洲有这样一个人。”老人想了片刻之后说道,那他应该是个非常有权势的人。“老人想了片刻之后说道,”就连部落的首领们需要的时候也不能见到他,只有当他自己作出决定时才行。

“等到仗打完之后,你们就随商队走吧,不要试图涉足绿洲的生活。”说完便离去了。

然而英国人却欣喜若狂,他们找对了路。

最后来了一位末穿黑色服装的少女,肩上扛着一只水罐,头上戴着面纱,但是脸却露在外面。男孩走上前去,向她问起炼金术士的事。

时间此刻仿佛静止不动了,世界灵魂全力地出现在男孩面前。当男孩望着少女的那双黑眼睛和在微笑与沉默之间迟疑不决的双唇时,他明白了世界所讲的语言中最重要和最智慧的那个部分,地球上所有的人心中都能理解的那个部分,这就是被称之为爱情的东西,它比人类和沙漠本身还更加古老,但是无论哪里有两双眼睛的目光相遇,就像现在一口水井前那两双眼睛的目光相遇一样,它都会以同样的伟力重新出现。少女的双唇终于决定露出一个微笑,这是一个预兆,一个男孩一生中不知道期盼了多么久的预兆,一个男孩曾在羊儿、书本、水晶制品和沙漠的沉寂那里寻找过的预兆。

这是世界上最纯正的语言,无需解释,因为宇宙无需解释便在无根的太空继续它的运行。此时此刻,男孩所能明白的一切就是来到了其生命中的那个女人面前,不需要任何言词,少女同样也应该明白这一点。男孩对此比对世上任何其他事物都更加坚信不疑,尽管他的父母和他父母的父母都说过,必须在恋爱、订婚、了解对方和有了钱之后才能结婚。说这种话的人也许从不懂得宇宙语言,因为假如懂得的话,就能很容易理解世界上总有一个人正在等待着另外一个人,无论是在沙漠之中还是在大城市里。当这样的两个人相遇而且目光交汇在一起时,所有的过去和所有的未来便都失去了其重要性,存在的只有这一时刻本身,还有不可思议地确信太阳底下所有的事情都是由同一只手写定的。这是一只唤醒爱情的手,是为在太阳底下工作、休息和寻找财宝的人们铸造了同样灵魂的一只手,如果没有这一切,人类的梦想便没有任何意义。

“马克图伯,”男孩想道。

英国人从坐着的地方站了起来,摇了摇男孩。

“快啊,快问她呀!”

男孩靠近少女,少女又微微一笑,男孩也微微一笑。

“你叫什么名字?”他问道。

“我叫法蒂玛。”少女望着地面答道。

“我们那个地方有一些女人也叫这个名字。”

“这是先知的女儿名字,”法蒂玛说道,“战士们把它带到了你们那里。”

“

娇弱的少女讲到战士时颇为骄傲。在一旁的英国人一个劲地催促,于是男孩问起此地是否有个能医治百病的男人。

“这个人通晓世上所有秘密,能与沙漠的精灵交谈。”少女说道。

精灵便是魔鬼。少女指了指南方,那位奇人就住在那边。

把水罐装满水之后少女就走了。英国人也走了,去找那位炼金术士。男孩在井边坐了很长的时间,他明白了某一天黎凡特风留在他脸上的正是这位少女的芳香,明白了甚至在知道她的存在之前就已经爱上了她,还明白了凭借他对少女的爱便能找到世界上所有的财宝。

先知系指伊斯兰教创始人穆罕默德(约570-632),他有个女儿名叫法蒂玛。[/face]

第二天,男孩又来到井边,等候着少女的出现。令他吃惊的是,他在那里遇到了英国人正在第一次凝视沙漠。

“我等了一个下午和一个晚上,”英国人说道,“直到最初的星星升起时他才露面。我向他讲述了我正在寻求什么,于是他问我是否已经把铅变成了金。我对他说,这正是我想要学会的东西。

“他让我试一试。去试试吧,这就是他对我所说的一切。”

男孩没有讲话。英国人长途跋涉,所听到的却是他已经知道的东西。男孩由此回想起,出于同样的原因,他给了老圣王六只羊。

“那就去试试吧。”他对英国人说道。

“这就是我正要去做的事,就从现在开始。”

英国人走后不久,法蒂玛就肩扛水罐前来汲水。

“我来只是想告诉你一件事情,”男孩说道,“我想让你成为我的妻子。我爱你。”

少女把水罐里的水放掉了。

“我每天都会到这里来等你。我穿越沙漠,想寻找到金字塔附近的一处财宝。对我来说,战争曾是一场灾难,现在却成了一种福音,因为它让我来到了你的身边。”

“战争总有一天会结束的。”少女说道。

男孩望了望绿洲里的椰枣树。他曾是个牧羊人,而这里有许多羊。法蒂玛比财宝更重要。

“战士们要寻找他们的财宝,”少女说道,仿佛猜出了男孩的想法,“沙漠的女人为她们的战士们感到骄傲。”随后她重新将水罐装满,接着便离去了。

男孩每天都去井边等候法蒂玛,把自己牧羊的经历、遇见圣王的事、在水晶店工作的情况都告诉了她。两个人成了朋友。除了与法蒂玛在一起的十五分钟之外,一天中其余的时间都漫长得似乎没有尽头。商队在绿洲停留了近一个月的时候,领队召集所有的人开了一次会。

“我们不知道战争会在什么时候结束,所以我们无法继续上路。”他说道,“战斗大概要持续很长时间,也许是许多年。交战双方都有强壮和勇敢的战士,两支部队都在为荣誉而战。这不是一场正义与邪恶的较量,而是为争夺同样的权力彼此间展开的一场战争。这种类型的战争一旦爆发,持续的时间要比其他战争更长,因为安拉同时站在交战双方两边。”

人们四散而去,那天下午男孩又与法蒂玛在井边相会,向她讲述了开会的情况。

“我们相识的第二天,”法蒂玛说道,“你向我吐露了你对我的爱情,后来你又教了我一些美好的东西,比如宇宙语言和世界灵魂。所有这一切,使我慢慢地成为了你的一个部分。”

男孩听着她说话的声音,认为比风吹动椰枣树时树叶发出的响声更加美妙动人。

“我在这个井边等了你很久了。我己经忘记了我的过去,忘记了传统,忘记了男人们所期望的沙漠女人应有的行为方式。从童年起,我就梦想沙漠会给我带来我生命中最大的礼物,这个礼物终于来了,那就是你。”

男孩想触摸少女的手,但法蒂玛的手正握着水罐的把手。

“你向我讲述了你所做的梦,老撒冷王,还有你的财宝。你还向我讲述了预兆。我什么都不怕,因为是这些预兆把你带给我的。我是你梦想的下个部分,是你常称作是你的天命的一个部分。

“因此我希望你朝着来时所追寻的方向继续前进。如果必须要等到战争结束,那很好。但是如果必须提前出发,那你就继续追寻你的天命去吧。沙丘会随风改变,而沙漠却永不改变。我们的爱情也是如此。

“马克图伯,”她最后说道,“假如我是你天命的一个部分,总有一天你会回来的。”

与法蒂玛分手后男孩感到十分忧伤。他回想起他认识的许多人来。结了婚的牧羊人总是很难说服妻子他们必须要外出到原野上去。爱情要求要与所爱的人厮守在一起。

第二天,他把这一切都告诉了法蒂玛。

“沙漠带走了我们的男人,而并不总是再把他们带回来。”法蒂玛说道,“所以我们己经习惯了。这些男人变得活在无雨的云朵之中,活在躲进石头之间的动物身上,活在从大地慷慨涌出的泉水里面。他们变成了万物的一部分,变成了世界灵魂。

“有些男人回来了,这时候,所有其他的女人也都十分高兴,因为她们所等待的男人同样有一天也可能回来。过去我望着这些女人,对她们所感到的快乐十分羡慕。如今,我也同样有了一个男人,期盼着他的归来。

“我是一个沙漠的女人,我为此感到骄傲。我希望我的男人像能移动沙丘的风那样自由行进。我同样也能在云朵之中、在动物身上和泉水里面看到我的男人。”

男孩去找英国人,想把法蒂玛的事告诉他。当他看到英国人在其帐篷旁边搭起一个小炉子时不禁大吃一惊。这个炉子颇为古怪,上面装有一个透明的细口小瓶。英国人正往火里添加木柴,并凝视着沙漠。他的双眼比把所有时间用于读书时似乎更为明亮。

“这是第一阶段的工作,”英国人说道,“我必须要把杂质硫磺分离出来。为做到这一点,我不能害怕失败。正因为害怕失败,时至今日我都一直没有尝试熔炼元精。现在我刚刚开始去做十年前就能够开始去做的事情。不过,我仍然感到高兴,因为我毕竟没有为此而等待二十年。”

接着他又继续往火里添加木柴,并凝视着沙漠。男孩在英国人身边呆了一段时间,直到沙漠开始被晚霞染成玫瑰红色。男孩此刻产生了一种强烈的愿望,他想到沙漠中去,看看沉寂能否解答他的疑问。

男孩随意地漫步了一段时间,始终没有让绿洲的椰枣树离开自己的视野。他倾听着风声,感觉着脚下的石块。有时候他会碰上某种贝壳,于是知道这片沙漠在过去某个遥远的年代曾是一片大海。后来他在一块岩石上坐下,任由眼前的地平线使自己进人被催眠状态。他不能理解没有占有欲的爱情,但法蒂玛是个沙漠的女人,如果有谁能教会她做到这一点,那肯定就是沙漠。

男孩就这样坐在那里,什么都不想,直至他首先感觉到远处有什么物体在他头顶的上方运动。他朝天空望去,看到有两只老鹰正在很高很高的上空飞翔。

男孩开始观察这两只老鹰和它们在空中飞出的图案。这些图案仿佛杂乱无章,然而对男孩而言却有着某种意义,只不过男孩还不能理解。于是他决定用目光追随它们的运动,也许他能读出某种东西。也许沙漠可以向他解释那种没有占有欲的爱情。

男孩开始发困。他的心灵要求他不要人睡:不仅不能人睡,而且要全神贯注。”我正在深入宇宙语言,世上的万物都有其意义,连老鹰的飞翔也是如此。”他自言自语道。他对一个女人充满了爱意,男孩借机对这件事报之以感谢。”人在相爱之时,万物会更加有其意义。”他想道。

突然,一只老鹰急速向下俯冲,对另一只鹰发动攻击。当老鹰作出这一动作时,男孩突然和急速地产生了一种幻景:一支拔剑出鞘的部队正进入绿洲。这一幻景虽然转瞬即逝,却令男孩惊慌不安。他曾听人说起过海市蜃楼,并亲眼目睹过几次,但那都是愿望在沙漠上的幻化成真,而他并没有让一支部队入侵绿洲的愿望。

男孩想忘掉这一幻景,重新回到自己的沉思之中。他试图把注意力再次集中到玫瑰红色的沙漠和岩石身上。但是,他内心深处的某种东西却使他无法平静。

“你要永远随预兆而行。”老圣王曾经说过。男孩想到了法蒂玛。他回忆起刚刚看到的幻景,预感到这件事很快就会发生。

男孩十分困难地从已经进入的思想状态中摆脱出来。他站起来,开始朝椰枣树方向走去。他再一次领会了事物的多种语言:这一次沙漠是安全的,而绿洲则变成了危险之地。

那位赶驼人坐在一棵椰枣树下,也在望着日落的景象。他看到男孩从一个沙丘后面走了过来。

“有一支部队正接近绿洲,”男孩说道,“我产生过一个幻景。”

“沙漠总是让一个人的心灵充满幻景。”赶驼人回答说。

男孩向赶驼人讲述了两只老鹰的事:当他突然深人世界灵魂之时看到了它们的飞翔。

赶驼人平静下来。他明白了男孩在说些什么。他知道,地球上任何一件东西都可以讲出万物的来龙去脉。无论是谁,如果把书随便翻到哪一页,或是观看人们的手相,或是通过扑克牌,或是通过观察鸟儿的飞翔,或是通过任何其他手段,都总会找到与自己正经历的事情相关的某种联系。其实并非是这些事物能够显示任何东西,而是那些观察这些事物的人发现了深入世界灵魂的方法。

沙漠到处都有这样的人,他们轻而易举地就能深入世界灵魂,并以此为生。这些人被称为占卜师,妇女和老人都对他们十分敬畏。战士们极少请他们占卜,因为如果知道什么时候会死,就不可能去投身于一场战斗。战士们宁愿体验拼杀的滋味,体验生死未卜的激动。未来已由安拉写定,而无论安拉所写定的是什么,都总是为了有益于人类。所以战士们只为现在而活着,因为现在充满着各种意外,他们必须要关注许多事情:敌人的剑在何处,马在哪里,下一次应该出何种招术才能保全生命。

赶驼人不是战士,他己请教过几位占卜师。他们中许多人的预言是正确的,另外一些人的预言则是错误的,直到有一天,他们中年纪最老(也最令人敬畏)的占卜师问赶驼人,为什么他对知道自己的未来如此感兴趣。

“为了能有所行动,”赶驼人回答说,“改变我不喜欢发生的那些事情。”

“那样一来,它们就不再是你的末来了。”老占卜师回答说。

“我想预知未来,也许是为了对将要发生的事情做好准备。”?

“假如是好事而你并不知道,那就将是一个意外的惊喜。”

占卜师说道,“假如是坏事而你己经知道,你就会在它发生之前忍受折磨。”

“我所以想预知未来,因为我是个人,”赶驼人对占卜师说道,“而人是靠着对未来的希望而活的。”

占卜师沉默了一阵儿。他是个玩掷签的行家:把签掷在地上,然后根据它们落地的形状作出解释。那一天他没有掷签,而是把它们包进一块手帕,又放进了衣袋里。

“我靠给人预测未来为生,”他说道,“我通晓掷签的学问,知道如何使用这门学问进入一切都已写定的那个空间,在那里我可以阅读过去,发现已被遗忘的事物,还能理解现实的预兆。

“当人们请教我的时候,我不是阅读末来,而是猜测未来。因为未来属于上帝,而只有他才在特殊的情况下向人们揭示末来。我是如何得以猜测未来的呢?是通过现实的预兆。奥秘就存在于现实之中。假如你关注现实,你就能够改善它。假如你改善了现实,那么将要发生的事情也就会变得更好。你要忘掉未来,按照教义过好你的每一天,并且要相信上帝会关顾他的孩子们。每一天的本身都蕴含着永恒。”

赶驼人很想知道,上帝在什么样情况下允许人们看到末来。

“当上帝本人想要揭示未来的时候。上帝极少揭示末来,这仅仅因为一个理由:这种未来已被注定要发生变化。”

上帝已向男孩揭示了一种末来,赶驼人想道,因为上帝希望这个男孩成为他的工具。

“去和部落的首领们谈谈,”赶驼人说道,“告诉他们有一支部队正在逼近。”

“他们会取笑我的。”

“他们是沙漠里的人,而沙漠里的人对预兆己习以为常。”

“那他们就应该己经知道。”

“他们并不关心预兆。他们相信,如果他们必须知道安拉想要告诉他们什么,就会有某个人来讲给他们听。这种事以前已发生过许多次。不过今天要讲给他们听的那个人就是你。”

男孩想到了法蒂玛。他决定去见部落的首领们。

“我带来了沙漠的预兆,”男孩对站在绿洲中央一座巨大 白色帐篷门口的卫兵说道,“我想见部落的首领们。”

卫兵什么话也没说,走进帐篷,在里面待了很长的时间。 后来他与一个年轻的阿拉伯人走了出来,后者身着白色带金 的服装。男孩向这位年轻人讲述了他所看到的景象。年轻人要男孩等一等,然后便又走进了帐篷。

夜色降临了。有几个阿拉伯人和商人走进和走出帐篷。一堆堆簧火逐渐地熄灭了,绿洲开始变得如同沙漠一般沉寂。只有大帐篷里的灯火依然亮着。在整个这段时间,男孩一直想着法蒂玛,仍然无法理解他们那天下午的谈话。

等待了许多个小时之后,卫兵终于让男孩进入了帐篷。

眼前的景象令男孩大为惊叹。他从未能想象到,在沙漠的中央竟然会有这样的一座帐篷。地面铺着他曾经踩踏过的最美丽的地毯,顶部悬挂着经过加工的黄金制成的灯具,上面点燃着蜡烛。部落的首领们成半圆形坐在帐篷最靠里边的地方,胳膊和腿放在带有精美绣饰的丝制垫子上面。仆人们进进出出,手里端着银制的托盘,上面放满了调味香料和茶水。有些人负责为水烟袋添加炭火。整座帐篷里弥漫着烟草的香味。

部落首领共有八位,但男孩马上便察觉出谁最为重要:一位坐在半圆正中身着白色带金服装的阿拉伯人。他的旁边坐着那个先前曾和男孩交谈过的年轻人。

“谁是那个谈论预兆的外国人?”一位首领望着男孩问道。

“我是。”男孩回答说,随后便讲述了他所看到的景象。

“沙漠知道我们几代人都生活在这里,为什么他要把这告诉给一个陌生人呢?”另一位首领发问说。

“因为我的眼睛对沙漠还没有习惯,”男孩回答说,“所以我能看到对沙漠过于习惯的眼睛不再能看到的东西。”

“还因为我了解世界灵魂,”男孩心中想道,但却没有说出口,因为阿拉伯人不相信这种事情。

“绿洲是中立地带,谁也不会攻击一个绿洲。”第三位首领说道。

“我只是把我看到的景象讲了出来,如果你们不愿相信,可以完全不去理会。”

帐篷里先是全然的沉默,接着部落首领们便热烈地讨论起来。他们讲的是阿拉伯语中的一种方言,男孩一点也听不懂,不过,当他表示要离去的时候,一个卫兵要他留下来。男孩开始感到有些害怕。预兆告诉他,某种事情出了问题,他后悔不该和赶驼人谈起这件事。

忽然,坐在中间的那位老者露出了一丝几乎察觉不到的微笑,男孩平静了下来。老者没有参与刚才的讨论,直到此刻为止还一言末发。然而男孩己熟悉了宇宙语言,所以能够感觉到帐篷里自始至终地弥漫着一种和平的气息。直觉告诉他说,他来对了。

讨论结束,众人静静地听着老者讲了一会儿话。老者随后转向男孩,脸上露出了冷峻而疏远的神情。

“两千年前,在一个遥远的地方,一个相信梦的人被投进井中和被卖作奴隶。”老者说道,“我们的商人把他买了下来,带回了埃及。我们这里的人都知道,相信梦的人也会解梦。”

“并非总能做到,”男孩想道,并回忆起那位吉卜赛老妇。“因为法老梦见了瘦牛和肥牛,这个人就把埃及从饥荒中解救了出来。此人名叫约瑟。和你一样,他也是来自一个外国的外国人,年龄大概也和你差不多。”

帐篷里继续悄无声息,老者的目光依然十分冷峻。”我们一直恪守传统。传统在那个时代把埃及从饥荒中解救出来,并使它成为最富有的民族。传统教导人们应该怎样穿越沙漠和怎样出嫁他们的女儿。传统规定绿洲是中立地区,因为交战双方都有自己的绿洲,它们都易于被攻破。”

老者讲话时没有任何人讲任何话。

“但是传统也告诉我们,要相信沙漠所给予的信息。我们所有的知识都是沙漠教给的。”

老者做了个手势,所有的阿拉伯人都站立起来。会议结束,水烟袋被熄灭,卫兵成立正姿势。男孩正准备离去,老者又对他说道:

“明天我们要打破任何人在绿洲都不能携带武器的约定。整个白天,我们都要准备迎击敌人。日落之时,要把武器交还给我。每逢有十名敌人被杀死,你就可以得到一个金币。

“但是,武器不能离开它的位置却不去经历战斗。它们和沙漠一样古怪任性。假如它们习惯于不被派上用场,那么下一次就可能会懒于进行发射。如果明天没有一件武器被派上”用场,那么至少有一件将会用在你的身上。”

男孩走出帐篷时,只有一轮圆月为绿洲照明。回到自己的帐篷要走上二十分钟,男孩迈动了脚步。

刚刚发生过的一切使他感到害怕。他深入进世界灵魂,而为了使人相信这一点所要付出的代价竟是他的生命。这个赌注的风险实在太高了。但是从卖掉羊群以追随天命的那一天起,他便已然下了风险极高的赌注。正如赶驼人所说的,明天死去如同任何其他一天死去一样都不是坏事。每一天的到来都是为了让人活着或是离开这个世界。一切都取决于一个词:“马克图伯”。

男孩默默地走着,并未感到后悔。假如他明天死去,那是因为上帝不想改变未来。但是在死去之前,他已经跨越了海峡,在一家水晶店打过工,了解了沙漠的沉寂,见识了法蒂玛的那双眼睛。从很久以前离开家乡之后起,他的每一天都过紧张而充实。哪怕是明天行将死去,他的眼睛所见到的事物也比其他牧羊人的要多得多。男孩为此而感到骄傲。

突然他听到一声巨响,接着便被一阵他从未见过的狂风猛地吹倒在地。尘埃四起,几乎遮住了月亮。一匹巨大的白马扬起前蹄直立在他的面前,发出了一声吓人的嘶鸣。

男孩儿乎看不清正在发生的事情,但是当尘埃稍稍落定之后,他却感到了一种过去从未有过的恐惧。一个全身着黑装的骑手跨坐在白马之上,左肩上立着一只猎鹰,头上缠着裹 头布,一条头巾遮住了整个的脸,只露出两只眼晴。看上去他好像是沙漠里传递消息的信使,但他的外表要比男孩一生中认识的所有人都更加强壮。

古怪的骑手从束在马鞍上的剑鞘里抽出一把巨大的弓形长剑,钢刃在月光的映照下闪闪发光。

“是谁竟敢解读老鹰飞翔的预兆?”骑手问道,声音如此洪亮,仿佛在绿洲的五万棵椰枣树中间激起了回响。 “我敢。”男孩说道,并立刻回忆起圣徒圣地亚哥?马塔莫罗斯的画像来:圣徒骑在他的白马上,一些异教徒倒在马蹄之下。眼前的情景恰恰正是如此,只不过位置颠倒了过来。

“我敢。”男孩又说了一遍,然后便低下头,等候着承接剑的刺击。

“许多人的生命将会被拯救,因为你们不懂世界灵魂。”

然而剑并没有飞快地刺下来。那个怪人的手慢慢地向下移动,直至剑锋碰到男孩的前额。剑锋无比锐利,男孩的前额流出了一滴血。

骑手全然地一动不动。男孩也同样如此。男孩连一分钟也没有想过要逃走,内心深处涌起了一种奇怪的快意:他将为自己的天命而死。还为法蒂玛而死。总之,预兆是真实的。敌人就在眼前,故而他无需担心死去,因为存在着世界灵魂,很快他就将成为它的一部分。明天敌人也将同样成为它的一个部分。

那个怪人只是继续将剑抵在男孩的前额。

“为什么你要解读老鹰飞翔的预兆?

“我只读了老鹰想要告诉我的东西。它们想拯救绿洲,你们将必死无疑。绿洲的人要比你们多。”

剑继续抵在男孩的前额。

“你是什么人,竟要改变安拉的旨意?”

“安拉创造了军队,也创造了鸟类。安拉向我展示了鸟类的语言。所有的一切都是同一只手写定的。”男孩说道,他想起了赶驼人的话。

怪人终于把剑从男孩的前额抽回。男孩感到了某种轻松。但他不能逃走。

“对预言要小心谨慎,”怪人说道,“事情一旦被写定,就无法加以避免。”

“我只看到了一支部队,”男孩说道,“并没有看到战斗的结果。”

骑手似乎对男孩的回答感到满意,但他仍然把剑握在手中。

“你一个外国人到一个外国来干什么?”

“追随我的天命。你是永远不会懂的。”

骑手将剑放进剑鞘,他肩上的猎鹰发出了一声怪叫。男孩开始放下心来。

“我需要试试你的勇气,”怪人说道,“对寻求宇宙语言的人来说,勇气是最重要的品质。” 男孩感到吃惊。这个人正在讲着只有很少人才了解的事情。

“尽管你己经走了很远的路,但必须永不松懈。”骑手接着说道,“你必须要热爱沙漠,但永远不要对它完全信任,因为沙漠对所有的人都是一场考验:它检验着人们所走的每一步,并杀死那些分心走神的人。”

怪人的话令人联想起老撒冷王的话。”如果有人前来进犯,而且日落之后你的头还长在脖子上面,你就来找我。”怪人说道。 那只曾握过剑的手现在握着一条马鞭。白马再次扬起前蹄,卷起一阵尘埃。

“你住在什么地方?”在骑手远去之时男孩高声喊问道。

握着马鞭的手指了一下南方。

男孩遇见的是炼金术士。

第二天早晨,两千名武装起来的男人埋伏在绿洲的椰枣树中间。太阳升到顶空之前,五百名战士出现在地平线上。他们骑马从北方进入绿洲,表面上看似是一次和平的进军,但是白色披风里面却隐藏着武器。当他们接近绿洲中央的大帐篷时,取出了弯刀和步枪,向一座空帐篷发起了攻击。

绿洲的男人们包围了这些沙漠骑兵。不到半个小时,四百九十九个骑兵便横尸在地。儿童们被安置在椰枣林的另一端,他们什么都没有看到。妇女们在帐篷里为她们的丈夫祈祷,也什么都没有看到。假如没有倒在地上的那些尸体,绿洲似乎和往日一样正常。

只有一名战士免于一死,此人便是这支队伍的指挥官。下午,他被带到部落首领们的面前,首领们质问他为什么要破坏传统。这位指挥官说,由于旷日持久的战斗,他们的人已经又饿又渴,精疲力竭,所以便决定攻占一座绿洲,以便能重新开始战斗。

部落主要首领说,他为这些战士们感到难过,但是传统永远不能遭到破坏。在沙漠里,惟一能够改变的只有随风而移动的沙丘。

随后他便判处这位指挥官以一种不体面的死刑。没有使用刀或是子弹,此人被吊死在一棵同样也已死掉的椰枣树上,任由尸体随着沙漠的风而摆动。

部落主要首领叫来了男孩,给了他五十枚金币。随后他又提及起约瑟在埃及的故事,请求男孩做绿洲的顾问。

太阳已完全沉落,第一批星星开始露面(它们并不十分明亮,因为圆月依然当空)。男孩朝南方走去。那里只有一座帐篷,几个路过的阿拉伯人对他说,这个地方到处都是精灵。但男孩还是坐了下来,等待了很长的时间。

当月亮己经高高升起之时,炼金术士来了,肩上挂着两只死去的老鹰。

“我来了。”男孩说道。

“你不应该来。”炼金术士答话道,“难道你的天命就是到此为止吗? ”

“部落之间发生了战争,使人无法穿越沙漠。”

炼金术士下了马,做了个手势让男孩随他进入了帐篷。这个帐篷与男孩在绿洲见过的其他所有帐篷一模一样——中央的大帐篷除外,它像童话故事里讲的那样豪华——男孩寻找着点铁成金所需要的器械和炉子,但却什么也没有发现。帐篷里只有不多的几本立放着的图书和一个做饭用的炉子,地毯上满是些神秘的图案。

“请坐,我去准备茶水,”炼金术士说道,“然后我们一起吃掉这两只鹰。”

男孩猜测它们就是他前一天看到的那两只鹰,但却什么也没有说。炼金术士点燃了炉火,帐篷里很快就满是一股肉的香味,比水烟袋的烟味要好闻多了。

“你为什么想要见我?”男孩问道。

“因为预兆的缘故。”炼金术士回答说,“风告诉我你将要到这里来,并且需要帮助。”

“那个人不是我,而是另外一个外国人,一个英国人,是他正在寻找你。”

“在找到我之前,他还必须经历其他一些事情。不过,他己走上了正路,开始观察沙漠了。”

“那我呢?”

“当你渴望得到某种东西时,整个宇宙都会协力使你实现自己的愿望。”炼金术士说道,重复了一遍老撒冷王的话。男孩明白了,另一个人来到了他的行进路上,引导他直达自己的天命。

“那么是你将要对我进行教导了?”

“不。你己经知道你所需要的一切。我只是要使你继续沿寻宝的方向前进。”

“部落之间发生了战争。”男孩重复了自己说过的话。

“我熟悉沙漠。”

“我已经找到了我的财宝。我拥有一头骆驼,还有在水晶店挣来的钱,还有五十枚金币。在我的家乡,我能够成为一个富翁了。”

“不过其中没有一样是来自金字塔附近的。”炼金术士说道。

“我有法蒂玛,她比我己经得到的所有这些财富都更加宝贵。”

“她同样也不是来自金字塔附近。”

两个人默默地吃了鹰肉。炼金术士打开一个瓶子,往男孩的杯子里倒了一种红色的液体。这是酒,是男孩有生以来喝过的最醇美的佳酿之一。但是当地的法律是禁止喝酒的。

“邪恶不是进入嘴的东西,”炼金术士说,“而是由嘴里出来的东西。”

酒开始使男孩感到快意,但是炼金术士却令他生畏。两个人坐在帐篷外头,望着使群星籍淡的明亮月光。

“喝吧,轻松愉快一下。”炼金术士说道,他发现男孩开始越来越开心了。”休息休息吧,就像战斗之前一个战士总要休息那样。但是不要忘记,你的心灵在哪里,你的财宝就在哪里。你的财宝需要被找到,以便使你在路上所发现的一切都能产生意义。

“明天你去卖掉你的骆驼,买一匹马。骆驼骗人,它们走上几千步也不露出任何疲劳的征兆,可突然间就会跪倒而死。马会渐渐地感到疲劳,所以你总能知道你能对它提出什么样的要求,或是它会在什么时候死去。”

第二天晚上,男孩牵着一匹马来到炼金术士的帐篷前。没等多久,炼金术士便骑着他的马来了,左肩上站立着那只猎鹰。

“把沙漠中的生命指给我看,”炼金术士说道,“只有发现生命的人,才能找到财宝。”

在依然明亮的月光照耀下,两个人开始沿着沙地前行。

“我不知道我能不能在沙漠中发现生命,”男孩想道,“我还不了解沙漠。”

男孩本想转身把自己的想法告诉炼金术士,但他对炼金术士望而生畏。两个人来到一片岩石地带,男孩曾在这里看到了天上的那两只老鹰,可现在此地却是一片沉寂,只有风声在响。

“我无法在沙漠中发现生命,”男孩说道,“我知道生命是存在的,但我却不能发现它。”

“生命吸引生命。”炼金术士回答道。

男孩明白了,他立刻松开了马的缰绳,任马自由地在岩石和沙土上行进。炼金术士默默地跟在后面。男孩的马走了几乎半个小时,此刻他们已经看不到绿洲的椰枣树了,有挂在天空的一轮巨大的月亮和闪烁着银色光芒的岩石。突然,在男孩从未来过的一处地方,他注意到马停止了脚步。

“这里存在着生命。”男孩对炼金术士说道,“我不熟悉沙漠的语言,但是我的马熟悉生命的语言。”

两个人下了马。炼金术士一言不发,开始一边打量岩石一边慢慢地行进。突然,他停下脚步,十分小心地弯下腰来。岩石间的地面上有一个洞,炼金术士把手伸了进去,接着直至肩膀的整个胳膊都伸了进去。洞里有个东西在动,炼金术士的眼睛——男孩只能看到他的眼睛——由于用力和紧张而收缩。他的胳膊似乎正在与洞里面的东西进行搏斗。只见炼金术士一跳——令男孩大吃一惊——,抽出了胳膊,并立刻站立起来。他的手紧紧抓住了一条蛇的尾巴。

男孩也跳了一下,只不过是向后而去。那条蛇在不停地挣扎,发出的咝咝响声打破了沙漠的沉寂。这是一条眼睛蛇,它的毒液能在短短的几分钟内致人于死命。

“小心它的毒液,”男孩想道。不过,炼金术士刚才曾经把手伸进过洞里,大概已经被咬过了,然而他的面部表情却很平静。“炼金术士己有两百岁了。”英国人曾经说过。他应该知道如何对付沙漠里的蛇。

男孩看到炼金术士来到马的身边,抽出长长的弯月形剑,用它在地上画了一个圆圈,然后把蛇放进圈的中央,蛇立刻安静下来。

“你放心好了,”炼金术士说道,“它不会出这个圈的。你发现了沙漠中的生命,这正是我所需要的预兆。”

“为什么这竟如此重要呢?”

“因为金字塔被沙漠所包围。”

男孩不想听人谈及金字塔。自前一天晚上开始,他的心情就沉重而忧郁,因为继续去寻找他的财宝意味着他不得不离开法蒂玛。

“我将在沙漠中为你领路。”炼金术士说道。

“我想留在绿洲。”男孩回答说,“我已遇到了法蒂玛,对我来说,她比财宝更珍贵。”

“法蒂玛是一个沙漠的女人,”炼金术士说道,“她知道,男人应该走出去,目的是为了能够回来。她己经找到了她的财宝,那就是你。如今她期盼着你能找到你所寻求的东西。”

“假如我决定留下来呢?”

“那你将成为绿洲的顾问。你会有足够的黄金去买许多羊和许多骆驼。你会与法蒂玛结婚,而且第一年你们将生活得十分幸福。你将学会热爱沙漠,会对五万棵椰枣树中的每一棵都了如指掌。你会观察到它们是如何成长的,如何显示出一个总在变化的世界。你还会越来越能理解预兆,因为沙漠是一位比所有老师都更好的老师。

“第二年,你会回想起有一份财宝的存在,预兆将开始不断地提及到它。而你则会试图对这些预兆显得一无所知。你只把自己的知识用来为绿洲和它的居民造福。部落的首领们会因此而感激你。你的骆驼将为你带来财富和权力。

“第三年,预兆会继续提及你的财宝和你的天命。你会几夜几夜地在绿洲上漫步,而法蒂玛将会成为一个闷闷不乐的女人,因为是她中断了你前进的道路。但是你会爱她,而且会得到回报。你会回想起她从未要求你留下来,因为一个沙漠女人懂得等待她的男人。因此你不会责怪她。但是你会有许多夜晚在沙漠的沙地上和椰枣树中间漫步,思付着当初自己也许应该继续前行,应该更加相信自己对法蒂玛的爱,你之所以留在绿洲,是因为你自己害怕再也不会回来了。到那个时候,预兆将向你昭示,你的财宝将永远被埋在地下。

“第四年,预兆将离你而去,因为你己经不想倾听它们的声音了。部落的首领们将会明白这一点,于是你将被解除顾问的职务。到那个时候,你将是个富有的商人,拥有许多骆驼和大量货物。但是你的余生将在椰枣树和沙漠之间度过,因为你将明白,你没有完成自己的天命,而现在再想这样去做已为时过晚。

“你永远不会明白,爱情从不阻止一个男人去追随他的天命。万一发生了这种情况,那是因为它不是真正的爱情,不是宇宙语言所讲的那种爱情。”

炼金术士把地上的圆圈去掉,那条蛇爬行而去,消失在岩石之间。男孩想起了总想去麦加朝圣的水晶店店主以及要寻找一位炼金术士的英国人。他还想起了一位姑娘,这位姑娘相信沙漠,有一天沙漠便给她带来了她盼望爱上的那个男人。

两个人上了马,这一次是男孩走在炼金术士的后面。风带来了绿洲的嗜杂声,男孩试图辨别出法蒂玛的声音来。那一天因为发生战斗的缘故,男孩没有到井边去。

但是在这个夜晚,当两个人注视着圆圈里的一条蛇时,肩上站立着猎鹰的怪异骑手曾谈论过爱情、财宝、沙漠的女人和他的天命。

“我将与你同行。”男孩说道,并立刻感到内心一片宁静。

“明天太阳升起之前我们动身。”这便是炼金术士的唯一回答。

男孩彻夜未眠。再过两个小时天就要亮了,他叫醒了同睡在一个帐篷里的其中一个少年,请他指出法蒂玛住在什么地方。两个人一起来到法蒂玛的帐篷前,作为回报,他给了这个少年一笔可以买上一只羊的钱。

然后他又让这个少年找到法蒂玛睡觉的位置,把她叫醒,告诉她他正在等她。阿拉伯少年照办了,因此又得到了一笔可以再买一只羊的钱。

“现在你让我们俩单独待上一会儿。”男孩对阿拉伯少年说道。阿拉伯少年回到自己的帐篷睡觉去了,他为能给绿洲顾问帮上忙而感到骄傲,也因为得到了可以买两只羊的钱而感到高兴。

法蒂玛出现在帐篷门口。两个人离开帐篷到椰枣树中漫步。男孩知道,他们的做法有违传统,但现在这却己毫不重要了。

“我要走了,”男孩说道,“我想让你知道我是会回来的。我爱你,因为……”

“什么都不必说了。”法蒂玛打断了他的话,“因为相爱,所以相爱,爱是不需要任何理由的。”

但是男孩还是继续说了下去:

“我所以爱你,是因为我做了一个梦,我遇到了一位圣王,我卖过水晶制品,我穿越了沙漠,部落之间发生了战争,我来到一个井边想打听一位炼金术士住在哪里。我所以爱你,是因为整个宇宙都协力使我来到你的身边。”

两个人拥抱在一起,这是他们第一次身体的接触。

“我会回来的。”男孩重复道。

“从前,我怀着祈求观望沙漠。”法蒂玛说道,“从现在起,我将怀着期盼观望沙漠。我的父亲有一天曾经离开过这里,但是他又回到了我母亲的身边,而且以后也总是继续回来。”

他们再也没有多讲什么。两个人在椰枣树中又走了一会儿,然后男孩把她送到帐篷门口。

“我会回来的,就像你父亲回到你母亲身边那样。”男孩说道。

他发现法蒂玛的眼晴里噙满了泪水。

“你哭了?”

“我是一个沙漠的女人,”法蒂玛藏起脸儿说道,“但我毕竟是个女人。”

法蒂玛走进了帐篷。没过多久,一轮旭日便冉冉东升。天亮之后,她走出帐篷,去做多年来一直所做的那些事情,但是一切都变得不同了。男孩已然不在绿洲,绿洲失去了直至不久前还具有的意义,不再是一个拥有五万棵椰枣树和三百眼水井的地方,不再是经过漫长的跋涉之后旅行者们兴高采烈到达的地方。从这一天起,对法蒂玛来说,绿洲将是一处空荡无物之地。

从这一天开始,沙漠将变得更加重要。她会一直望着沙漠,试图知道男孩正追随着哪一颗星星去寻找财宝。她将会通过风送去她的亲吻,期盼着风能吹拂到男孩的脸,并告诉男孩她还活着,正犹如一个女人等待着一个勇敢的寻找梦想和财宝的男人那样等待着他的归来。从这一天起,沙漠无非只意味着一件事:

盼望他的归来。

“不要去想己经过去的事情。”当他们开始骑马在沙漠上行走时,炼金术士说道,”一切都已铭刻在世界灵魂上面,并且在那里永存。”

“人们梦想归来要多于离去。”男孩说道,他已经再次习惯沙漠的沉寂。

“如果你所遇到的是纯物质制成的东西,那它就永远不会腐朽。总有一天你会回来的。如果你所遇到的只是瞬间的光芒,如同一颗星星爆裂似的,那么当你回来的时候将一无所获。但是你毕竟见到了光的爆裂,仅此一点也就值得了。”

他讲的是炼金术的用语,但是男孩明白他指的乃是法蒂玛。

不去想过去的事情是很难做到的。几乎总是一成不变的沙漠景象常使人充满幻觉。男孩依然能看到椰枣树、水井以及他所爱着的那个女人的相貌。他看到了正在实验室里的英国人,还有那位赶驼人。赶驼人是位导师,但他自己却并不知道。“也许炼金术士从未恋爱过,”男孩想道。

炼金术士骑马走在前头,肩上站着猎鹰。猎鹰十分精通沙漠的语言,当他们停下来时,猎鹰便离开炼金术士的肩头,飞去寻找食物。第一天它带回一只兔子,第二天带回两只鸟。

入夜之后,他们铺开毛毯,并不点燃篝火。沙漠的夜晚十分寒冷,且随着天空的月亮开始缩小而变得黑暗。他们默默无语地走了一个星期,只就采取必要的戒备以避开部落之间的战争交谈过几句。战争仍在继续,有时风会带来血的淡淡的甜味。附近曾发生过一场战斗,风提醒男孩预兆语言的存在,它随时准备显现出男孩的眼睛无法看到的东西。

第七天的旅程结束时,炼金术士决定要比往常早一些露营。猎鹰飞出去寻找猎物,炼金术士取出旅行水壶递给了男孩。

“现在你几乎到达旅程的终点了,”炼金术士说道,“我祝贺你追随了自己的天命。”

“你一直默默无语地为我领路,”男孩说道,“我本以为你将把你所会的东西教给我。前一段时间,我在沙漠中与一个男人在一起,他带有一些炼金术的书籍,可我什么也没有学到。”

“只有一种学到的方式,”炼金术士回答说,“那就是通过行动。你所需要学到的一切,旅行都已教会了你,现在就只差一件事情了。”

男孩想知道是什么,但是炼金术士的眼睛却注视着地平线,等候着猎鹰的归来。

“为什么人们把你叫作炼金术士?”

“因为我是炼金术士。”

“其他的炼金术士们炼金却没有成功,他们错在什么地方呢?”

“他们只是炼金而已,”炼金术士回答道,“他们寻找自己天命中的财宝,却不希望履行自己的天命。”

“我还需要学会什么呢?”男孩问道。

但炼金术士继续注视着地平线。过了一会儿,猎鹰带着食物回来了。他们挖了一个洞,在洞里点燃了火,以避免任何人能见到火光。

“我所以是炼金术士,因为我就是炼金术士。”当两个人准备他们的饭食时炼金术士说道,”我是从我祖父那儿学会的,我的祖父是从他的祖父那儿学会的,以此类推,一直上溯到创世之初。在那个时代,有关元精的全部知识可以写在一块简简单单的绿宝石书板上。但是人们看不起简单的事物,于是便开始著书立说,进行论释和做哲学研究,并且开始说自己比别人更通晓此道。

“然而那块绿宝石书板却一直流传至今。”

“绿宝石书板上写了些什么呢?”男孩很想知道它的内容。

炼金术士开始在沙地上画了起来,不到五分钟便画完了。在他画的同时,男孩想起了老撒冷王以及他们相遇的那个广场。这件事仿佛已过去了许多许多年。

“这就是写在绿宝石书板上的东西。”画完之后炼金术士说道。

“是密码。”男孩说道,他对绿宝石书板感到有些失望,“与英国人的那些图书很是相似。”

“不,”炼金术士回答道,“它如同老鹰们的飞翔,不应该简单地通过理智去理解。绿宝石石板是直接通往世界灵魂之路。

“智者们都懂得,这个自然的世界只不过是天堂的一种影像和复制品。这个世界的简单存在,只是肯定还存在着一个比它更加完美的世界。上帝创造它的目的在于,通过视力可见之物,人类能够理解他的精神教诲以及他的智慧奇迹。我把这称为‘行动作用’。”

“我应该弄懂绿宝石书板吗?”男孩问道。

“也许吧。假如你现在是在一间炼金术实验室里,此刻将会是研究理解绿宝石书板最佳方法的最好时机。但是你现在是在沙漠里,所以你要深入到沙漠中去。沙漠如同世界上任何其他东西一样,将有利于你去理解世界。你甚至无须理解沙漠,只要你静观一粒简单的沙子,你就能从中看到天地万物的全部神奇所在。”

“我怎么才能深人到沙漠中去呢?”

“倾听你的心灵。它知道所有的事物,因为它来自世界灵魂,而且总有一天还会回到那里去。”

两个人默默无语地又走了两天。炼金术士变得更加小心谨慎,因为他们正接近战斗最为激烈的地区。男孩则力图倾听自己的心灵。

这是一颗难以相处的心灵。从前它总是习惯于出发前进,现在则是竭尽全力想要回归。有时他的心灵几个小时地讲述着离愁别绪的故事,有时又被沙漠的日出深深感动,并使男孩偷偷地流泪。当向男孩讲起财宝一事时,它的跳动就会加快;而在男孩出神地凝视沙漠无根的地平线时,它的跳动就会变缓。但它永无平静之时,即使在男孩与炼金术士互相不讲一句话时也是如此。

“为什么我们必须要倾听心灵呢?”那一天当他们露营时男孩问道。

“因为心灵所在之处,正是你的财宝所在之地。”

“我的心灵很不平静,”男孩说道,“它有梦想,容易激动,热恋上沙漠的一个女人。当我思念她的时候,心灵便要求我做这做那,许多夜晚都不让我人睡。”

“这是好事,说明你的心灵是活跃的。你继续倾听它有什么事情要对你说。”

在以后的三天里,他们俩曾从几名部落战士身边走过,并看到在地平线上还有另外一些部落战士。男孩的心灵开始谈论恐惧。它向男孩讲述了它从世界灵魂那里听来的故事:有些人去寻找自己的财宝,但却永远没有找到。有时候,心灵用男孩可能找不到自己的财宝或是可能会死在沙漠的念头恐吓男孩。另一些时候,心灵又对男孩说它已经很满足,己经找到了爱情和有了许多金币。

“我的心灵对我不忠。”当两个人停下来让马歇息片刻时,男孩对炼金术士说道,“它不想让我继续前进。”

“这是好事,”炼金术士回答道,“证明你的心灵是活跃的。为了一个梦而失去己经得到的一切,有这种担心是很自然的。”

“那我为什么要倾听我的心灵呢?”

“因为你永远无法让它沉默。哪怕是你装作不听它所讲的东西,它也依然总是在你的心底反复陈述它对生活和世界的想法。”

“即使是它对我不忠?”

“它的不忠是你未曾预料到的打击。假如你很好地了解你的心灵,它就永远做不到这一点,因为你会了解它的梦想和希望,并且知道如何为之而努力。

“谁也无法逃避自己的心灵,因此最好是倾听它所讲的东西,这样的话,你预料不到的打击就永远不会发生。”

当他们在沙漠中行进的时候,男孩继续倾听着自己的心灵。他开始了解它的诡计与花招,并按其本来面目接受了它。于是男孩不再感到恐惧,也不再想要返回绿洲,因为一天下午他的心灵对他说它很快乐。“尽管我有一些抱怨,”他的心灵说道,”那因为我是一个人的心灵,而人的心灵就是这个样子。它们害怕实现自己更大的梦想,因为它们认为自己不配有这样的梦想,或是认为不能实现这样的梦想。一想到一去不再复返的恋人,一想到本来可以是美好的时刻而实际上却并非如此,一想到本来可以发现的财宝却永远被埋藏在沙地下面,我们心灵就怕得要死,因为当这些情况发生时,我们终将要忍受极大的痛苦。”

“我的心灵畏惧忍受痛苦。”一天晚上,当两个人望着没有月亮的天空时,男孩对炼金术士说道。

“告诉它,畏惧忍受痛苦比忍受痛苦本身更加糟糕。没有一个心灵在追寻它的梦想时会忍受痛苦,因为追寻中的每一刻都是与上帝和永恒相遇的时刻。”

“追寻中的每一刻都是与上帝和永恒相遇的时刻。”男孩对他的心灵说道,“在我寻找我的财宝的过程中,每一天都是光辉灿烂的,因为我知道,其中的每一个小时都是实现梦想的一个组成部分。在我寻找我的这份财宝的过程中,一路上发现了我过去从未梦想过会遇到的东西,假如当初我没有勇气去尝试那些对牧羊人来说是不可能的事物,这一切就都将无从谈起。”

于是他的心灵平静了整整一个下午。人夜之后,男孩睡得很安稳。醒来时,他的心灵开始向他讲述起世界灵魂的事情来。心灵说,一个心中有上帝的人是十足幸福的。正如炼金术士所讲的那样,可以在沙漠的一粒普通的沙子上找到幸福,因为一粒沙子便是创世中的一个时刻,为了创造它,宇宙要花费数十亿年的时间。”世上的每一个人都有一份财宝在等候着他,”男孩的心灵说道,”而我们心灵通常很少谈起这些财宝,因为人们已经不再想找到它们。我们只对孩童谈起它,然后我们便让生命各自朝它天命的方向行进。不幸的是,只有少数人沿着为他们规定的道路前行,而这乃是通向天命之路,通向幸福之路。人们把世界视为一种具有威胁之物,正因为如此,世界就变成了一种具有威胁之物。

“于是我们心灵讲话的声音越来越小,但却永远不会沉默。我们屈从退让,不便我们的话语被听到,因为我们不想人们因为不听从心灵的话语而忍受痛苦。”

“为什么心灵不告诉人们应该继续追寻他们的梦想呢?”男孩向炼金术士问道。

“因为在这种情况下,心灵会忍受更大的痛苦,而心灵不喜欢忍受痛苦。”

从那一天起,男孩理解了自己的心灵。他请求心灵永远都不要离开他。他还请求,当他远离自己的梦想时,心灵要在胸膛加快跳动,发出警报的信号。男孩发誓,只要一听到这个信号,便会听从它的指引。

那天晚上,男孩把这一切都告诉了炼金术士。炼金术士明白,男孩的心灵己返回世界灵魂。

“现在我该怎么做呢?”男孩问道。

“朝金字塔方向前进。”炼金术士说道,“你要继续留意预兆。你的心灵己经能够为你指出财宝在何处了。”

“这就是我所需要知道的吗?”

“不。”炼金术士回答说,”你需要知道的事情如下:

“在实现一个梦想之前,世界灵魂总是决定要对寻梦者一路所学到的一切加以检测。它所以这样做并非因为它居心不良,而是让我们与梦一起,也能够掌握在寻梦过程中所学习到的知识。这是多数人会放弃的一个时刻。在沙漠的语言中,我们把这称之为‘渴死在椰枣树己经出现在地平线上之时’。

“每一次寻梦都以创始者的运气开始,又总以对征服者的考验结束。”

男孩回想起故乡的一句古老的谚语:最黑暗的时刻出现在黎明之前。

翌日,第一个具体的危险信号出现了。三名部落战士走近他们,询问男孩和炼金术士到这里来干什么。

“我是带着猎鹰来狩猎的。”炼金术士回答道。

“我们需要对你们进行搜查,看看你们是不是带着武器。”其中的一位战士说道。

炼金术士缓缓地下了马,男孩也同样如此。

“你带着这么多钱干什么?”搜查男孩钱袋的战士问道。

“为了能到达埃及。”男孩回答说。

搜查炼金术士的战士发现了一个装满液体的小水晶瓶,还有一个比鸡蛋略大一点的黄色玻璃蛋。

“这是些什么东西?”战士问道。

“哲人石和长命液,是炼金术士们炼出的元精。谁要是喝了这种液体,谁就永远不会生病;而这块石头的一点点碎片就能把任何金属变成黄金。”

战士们哈哈笑了起来,炼金术士也和他们一起笑了。这些战士认为炼金术士的回答十分有趣,于是末加阻拦就放两个人带着他们的全部东西离去了。

“你疯了吗?”当他们走出相当远的时候,男孩向炼金术士问道,“你为什么要那样说呢?”

“为了向你说明世上的一个简单规律。”炼金术士回答说,“当我们眼前拥有巨大的财宝时,我们却永远不会察觉。你知道为什么吗?因为人们不肯相信这些财宝的存在。”

两个人继续在沙漠申行进。随着时间一天天过去,男孩的心灵就变得日益沉默。它已然不想知道过去的事情或是将来的事情,而是也满足于观察沙漠奇-[书]-网,与男孩一起从世界灵魂中汲取东西。男孩和他的心灵成了好朋友,彼此都无法背叛对方。

心灵开口讲话,乃是为了给男孩以鼓舞和力量,因为有时候男孩觉得沉默无语的日子会令人感到极大的厌烦。心灵第一次告诉了男孩他所具有的伟大品质:放弃羊群和追寻天命时所表现出的勇气,在水晶店工作时所表现出的热情。

心灵还告诉男孩一件男孩过去未曾注意到的事情:危险曾与他擦肩而过,但男孩却从末察觉到。心灵说,有一次,它把男孩从父亲那里偷来的一支手枪藏了起来,因为这支枪很可能会伤害到男孩自己。它还回忆说,有一天,男孩在野外生了病,呕吐起来,然后就睡了很长的时间。当时有两个强盗正在前面等候他,打算抢走他的羊,并且把他杀害。但是因为男孩没有来,他们便以为男孩改变了路线,于是就离去了。

“心灵总是给人以帮助吗?”男孩问炼金术士。

“只帮助那些追随自己天命的人。不过,它们更肯帮助儿童、醉汉和老人。”

“这意味着我不会有危险吗?”

“只意味着心灵会全力以赴。”炼金术士回答道。

一天下午,他们路经一个部落的营地。营地的所有角落都有阿拉伯人,他们穿着醒目的白色服装,装备着武器,一边吸着水烟袋,一边讲着战斗的故事。谁都没有太在意两个路过的旅行者。

“没有任何危险。”离开营地还不远时男孩说道。

炼金术士生气了。

“要信任你的心灵,”他说道,“但是也不要忘记你正身处沙漠之中。当人们进行战争时,世界灵魂同样能感受到战斗的呐喊声。谁也不能不承受太阳底下发生的每一件事所造成的后果。”

“万物为一,”男孩想道。

仿佛沙漠有意要证明老炼金术士的话是正确的,在他们的身后出现了两位骑兵。

“你们不能再往前走了,”其中的一位骑兵说道,“你们来到了正在打仗的地区。”

“我不会走出很远。”炼金术士回答说,同时直视着骑兵们的眼睛。两位骑士沉默了片刻,随后便同意他们俩继续前行。

男孩出神地目睹了这一切。

“你用目光制服了他们。”他说道。

“眼睛能够显示灵魂的力量。”炼金术士回答道。

“千真万确,”男孩想道。他已经察觉到,在营地的那群战士当中,有一个人正紧紧地盯着他们俩,由于距离太远,所以男孩无法看清这个人的面孔。但是男孩确信,此人正在注视着他们。

当他们开始翻越绵延于整个地平线的一座山脉时,炼金术士对男孩说,再有两天他们就可以抵达金字塔了。

“如果我们马上就要分手的话,”男孩说道,“就请你教教我炼金术吧。”

“你已经会了,那就是深人到世界灵魂中去,发现它为我们保留的财宝。”

“这不是我想知道的,我说的是如何把铅变成黄金。”

炼金术士尊重沙漠的沉默,直到他们停下来准备吃饭的时候,才回答了男孩。

“宇宙万物都在演化,”他说道,“对智者而言,金子是演化最甚的金属。你不要问为什么,因为我也不知道。我只知道传统总是正确的。

“世人没有正确地领会智者的话,于是金子不仅没有成为演化的象征,反而成为战争的信号。”

“事物会讲许多种语言,”男孩说道,“当骆驼嘶鸣时,我只把它看成是一声嘶鸣,后来它却变成了提示危险的信号,最终又重新变成一声嘶鸣。”

男孩止住了自己的话。炼金术士应该知道所有这一切。

“我认识一些真正的炼金术士,”炼金术士接着说道,”他们把自己关在实验室里,努力使自己像金子一样发生演化,于是便发现了哲人石,因为他们懂得,当一种东西发生演化时,它周围的一切也都会发生演化。

“还有一些炼金术士偶然得到了哲人石。这些人具有天赋,他们的灵魂比其他人的更为清醒。但是这些人不能算数,因为他们为数太少。

“最后还有一些炼金术士,他们所追求的只是金子而已,所以永远不能发现其中的奥秘。他们忘记了,铅、铜、铁同样也有自己要履行的天命,谁干涉其他事物的天命,谁就永远发现不了自己的天命。”

炼金术士的话语仿佛是一个诅咒发出了声音。他弯下身,在沙地上捡起一个贝壳。

“这个地方从前有一天曾是海洋。”炼金术士说道。

“我已经注意到了。”男孩回答说。

炼金术士让男孩把贝壳放到耳边。小的时候,男孩曾这样做过许多次,这一回他又听到了海的喧闹。

“大海继续留在这只贝壳里,因为这是它的天命。它永远不会离开贝壳,直到沙漠重新被海水覆盖。”

两个人随后骑上马,继续朝金字塔的方向行进。

太阳开始西沉时,男孩的心灵发出了危险的信号。此刻,他们正身处一些巨大的沙丘中央,男孩望了炼金术士一眼,但炼金术士仿佛什么都没有注意到。五分钟之后,男孩发现自己面前出现了两名骑兵,阳光照出了他们的侧影。末等男孩对炼金术土讲话,两名骑兵就变成了十名,接着又变成了一百名,最后则布满了所有的巨大沙丘。

这些人身着蓝色服装,缠头布上套着黑色的头圈,蓝色的纱巾遮着面孔,只露出了眼睛。

虽然离得甚远,依然能看出他们的眼睛显示出了他们灵魂的力量。这些眼睛正在讲述着死亡。

两个人被带到附近的一座军营。一名士兵把男孩和炼金术士推进了一座帐篷。这座帐篷与男孩在绿洲见到过的帐篷不同,一名指挥官正在里面与他的参谋人员开会。

“他们是间谍。”其中的一个人说道。

“我们只是旅行者。”炼金术士回答说。

“三天前,有人看到你们在敌方的营地里,你们还和他们中的一名战士谈过话。”

“我是个在沙漠里四处走动的人,我了解星象。”炼金术士说道,“我对军队,或是说部落的行动一无所知。我只是为给我的朋友当向导才来到这里。”

“你的朋友是什么人?”指挥官问道。

“一名炼金术士。”炼金术士回答说,“他通晓自然界的威力,并希望向指挥官展示他的特异功能。”

男孩静静地听着,心里十分害怕。

“一个外国人跑到一个外国来干什么?”另外一个人问道。

“他带来了钱想献给你们部落。”不等男孩开口,炼金术士便抢先回答说,接着就拿过男孩的钱袋,把金币交给了指挥官。

阿拉伯人不声不响地接了过去。这些金币可以购买许多武器。

“炼金术士是干什么的?”最后他问道。

“炼金术士通晓自然和世界的奥秘,假如他愿意的话,只要借助风的力量就能把这个营地摧毁。”

那些人笑了起来。他们熟知战争的力量,风并不能阻止一次致命的攻击。然而在他们每个人的胸膛,心脏都加快了跳动。他们是沙漠的男人,对巫师充满恐惧。

“我倒想看一看。”指挥官说道。

“我们需要三天的时间。”炼金术士回答说,“他会把自己变成风,目的不过是要展示一下他所掌握的功能的威力。倘若他做不到的话,我们将卑躬屈膝地为贵部落的荣誉而贡献我们的生命。”

“你用不着把己经属于我的东西贡献给我。”指挥官趾高气扬地说道。

不过,他同意给他们三天的时间。

男孩吓得全身不能动弹。炼金术士抓住他的胳膊,把他架出了帐篷。

“不能让他们察觉出你的恐惧。”炼金术士说道,“他们都是些勇敢的人,鄙视胆小鬼。”

但是男孩却说不出活来。过了一段时间,当他们行走到营地中央时,男孩才终于能够开口讲话了。没有必要把他们关押起来,阿拉伯人仅仅牵走了他们的马。世界再次展示了它的多种语言:此前,沙漠还是一片自由和无根的天地,现在却成了一道无法逾越的大墙。

“你把我所有的钱全给了他们!”男孩说道,“这是我整整一生挣来的!”

“假如你难免一死的话,这些东西对你又有什么用处呢?”炼金术士回答说,“你的钱救你能再活上三天。钱能用来推迟一个人的死期,这种情况并不多见。”

但是男孩正处于极度的恐惧之中,所以根本无法听进这些充满智慧的话语。他不知道如何把自己变成风。他不是一位炼金术士。

炼金术士向--位战士要来了茶水,倒了一点在男孩的手腕上。一阵宁静的潮水涌满了男孩的全身。与此同时,炼金术士讲了一些男孩末能听懂的话语。

“你不要沉溺于绝望之中不能自拔,”炼金术士说道,声音显得出奇地柔和,“这样会使你无法与你的心灵进行交谈。”

“可我不知道怎么把自己变成风。”

“谁正追寻自己的天命,谁就能知道他所需要知道的一切。只有一样东西能使梦想无法成真,那就是对失败的担心。”

“我并不担心失败,我只是不知道怎么把自己变成风。”

“你必须要学会。你的生死取决于此。”

“万一我做不到呢?”

“那你就会在追寻自己的天命过程中死去。这比像数以百万计的人那样死去不知要好多少,因为那些人根本不知道有天命的存在。

“但是你不必担心,死亡通常能令人对生命更有感情。”

第一天过去了。附近发生了一场大规模战斗,有几位伤员被送回了军营。“一切都不因为死亡而发生变化,”男孩想道。死去的战士的位置被其他战士所取代,生活照旧继续进行。

“你本来可以晚一点再死去,”一个战士对他一个同伴的尸体说道,”本来可以等到和平降临的时候再死去。不过,无论如何,你最后总是要死的。”

傍晚时分,男孩去找炼金术士。炼金术士正要带着猎鹰去沙漠。

“我不知道怎么把自己变成风。”男孩重复道。

“记住我对你说的话:世界只是上帝可以被看到的那个部分。炼金术士就是要把精神的完美带到物质的结构。”

“你在干什么?”

“喂我的猎鹰。”

“万一我不能把自己变成风,我们俩都得死去,”男孩说道,“为什么还要喂猎鹰呢?”

“要死的是你,”炼金术士说道,“我知道怎么把自己变成风。”

第二天,男孩爬上军营附近一块巨大的岩石。哨兵们允许他通行,他们己经听说他是能把自己变成风的巫师,所以不愿靠近他。何况沙漠是一座无法逾越的大墙。

整个一个下午,男孩一直凝视着沙漠,倾听自己的心灵。沙漠听出了他的恐惧。

男孩和沙漠讲着同一种语言。

第三天,指挥官召集来他的主要的军官。

“我们去看看那个要把自己变成风的男孩。”他对炼金术士说道。

“走吧。”炼金术士回答说。

男孩把他们领到他前一天去过的地方,并请他们全都坐下来。

“要花费一点时间的。”男孩说道。

“我们不急,”指挥官说,“我们是沙漠的人。”

男孩开始向前方的地平线望去。远处是连绵起伏的群山。沙丘、岩石随处可见,还有爬行植物,在无法生存之地顽强地生长着。眼前便是沙漠,他已经在上面行走了好儿个月之久,但即使如此,他也只熟悉其中的一个极小部分。正是在这个极小的部分,他遇到了英国人、商队和部落之间的战争,还有一个拥有五万棵椰枣树和三百口水井的绿洲。

“你今天到这里来想做什么?”沙漠问道,“昨天我们不是相互观望得足够了吗?”

“你把我所爱的一个人保留在某个地方,”男孩说道,“于是,在我望着你的沙粒时同样也望到了她。我想回到她的身边,需要你帮助把我变成风。”

“爱是什么?”沙漠问道。

“爱就是猎鹰在你的沙粒上空飞翔的时候,因为对它而言,你就像一片绿地,它永远不会无功而返。它熟悉你的那些沙丘、岩石和群山,你对它是十分慷慨的。”

“猎鹰的嘴从我身上叼去了部分东西。”沙漠说道,“我穷年累月地培育它的猎物,用我不多的一点点水喂养它们,向它们指出食物所在的地方。于是有一天,当我正要感受猎物在我的沙粒上进行爱抚时,猎鹰就从天而降,把我喂养的东西叼走。”

“可你正是为此而培育猎物,”男孩说道,“目的是为了喂养猎鹰。猎鹰将给人带来食物,而人有一天又会为你的沙粒供给养分,猎物将重新在那里出现。世界就是这样运行的。”

“这就是爱吗?”

“这就是爱。是爱使猎物转化成猎鹰,猎鹰转化为人,人又重新转化为沙漠。是爱使铅变成了金子,而金子又重新藏身于地下。”

“我不明白你说的话。”沙漠说道。

“那么请你明白,在你沙粒的某个地方,有一个女人在等待着我,为此我必须要变成风。”

沙漠沉默了一段时间。

“为了让风刮起来,我可以把我的沙粒给你。但我无法独自做成任何事情。你求风来帮忙吧。”

一阵微风开始刮了起来。部落的军官们从远处望着男孩,男孩所讲的是一种他们听不懂的语言。

炼金术士微笑起来。

风靠近男孩,触摸了他的脸。风听到了刚才男孩与沙漠的谈话,因为风总是无所不知。风吹遍整个世界,没有诞生之所,也没有消亡之地。

“请帮助我。”男孩对风说道,“有一天,我曾从你那里听到了我所爱的人的声音。”

“是谁教会你能讲沙漠和风的语言?”

“我的心灵。”男孩回答说。

风有许多名字。在这个地方,它被叫作西罗科,因为阿拉伯人相信,它来自被水覆盖的土地,那里居住着黑人。在男孩遥远的故乡,它被称为黎凡特,因为人们相信它带来了沙漠的沙粒和摩尔人战斗的呐喊声。也许在一个更为遥远的放牧羊群的田野,人们会以为风起于安达卢西亚。然而风并不起于任何一个地方,也不止于任何一个地方,因此它比沙漠更为强大。将来有一天,人们能够在沙漠里种上树,甚至可以在那里养羊,但却永远不能够控制住风。

“你无法成为风,”风说道,“我们的本性不同。”

“并非如此,”男孩说道,“当我与你漫游世界时,我掌握了炼金术的秘密。风、沙漠、海洋、星星以及宇宙中的一切造物,在我的身上应有尽有。我们都是由同一只手创造出来的,我们拥有同样的灵魂。我希望如同你一样,进入到所有的角落,跨越海洋,清除掉掩盖着我的财宝的沙子,把我所爱的女人的声音带到我的身边来。”

“那一天我听到了你与炼金术士的谈话,”风说道,“他说,每样事物都有自己的天命。人不可能变成风。”

“请你教我成为一会儿风吧,”男孩说道,“以便我们能够就人和风的无限可能性进行交谈。”

风很好奇,这是一件它不曾了解的事情。它很愿意跟男孩探讨这个话题,但却不知道如何把人变成风。请看风经历过多少事情啊!它建造过沙漠,沉没过船只,摧毁过整座森林,穿越过充满音乐和奇怪噪音的城镇。风认为自己是不受限制的,可现在这个男孩说,还有更多的事风能够做到。

“这就是人们所说的爱。”看到风几乎要答应自己的要求时,男孩说道,“当我们在爱的时候,就可以成为天地万物中的任何一种。当我们在爱的时候,就根本没有任何必要去弄懂所发生的事情,因为一切都发生在我们的心灵深处,而人是可以变成风的。当然,这只有风肯帮忙才行。”

风十分傲慢,男孩的话刺激了它。于是它开始更猛烈地刮了起来,扬起了沙漠的沙粒。然而最终风不得不承认,尽管它跑遍了整个世界,也不知道如何把人变成风。它不了解爱。

“当我在世界各地漫游时,发现许多人谈起爱的时候都仰望着天空,”风说道,它因为不得不承认自己的局限性而怒气冲冲,“也许最好是问问天空。”

“那就请你帮助我,”男孩说道,“让这个地方充满尘埃,使我能够观望太阳而不会被它弄瞎了眼睛。”

于是风更加用力地刮了起来,天空立刻布满了沙尘,使太阳所在之处仅仅剩下一个金黄色的圆盘。

军营里变得难以看清东西。沙漠的人熟悉这种风。人们称它为西蒙,比海上的暴风雨更可怕——因为他们没见过海。马儿开始嘶叫,武器被沙尘所覆盖。

一位军官在岩石上转向指挥官说道:

“也许最好到此为止。”

这些人己经看不到男孩了。他们的脸被蓝色头巾遮住,眼睛则只充满恐惧。

“我们到此为止吧。”另一位军官也说道。

“我想看看安拉的伟大,”指挥官尊敬地说道,“我想看看人如何变成风。”

他在心里记下了那两位感到恐惧的军官的名字。这阵风一停下来,他就将解除他们的职务,因为沙漠的人要无所畏惧。

“风对我说你了解爱。”男孩对太阳说道,“如果你了解爱,你也就了解世界灵魂,因为它是由爱形成的。”

“从我所在的位置,”太阳说道,“我可以看到世界灵魂。它与我的灵魂相通,我们一起使植物生长,让羊儿去寻找荫凉。从我所在的位置——我距离地球十分遥远——,我学会了爱。我知道,假如我再靠近地球一点,地球上的万物就会全部消亡,世界灵魂就将不复存在。所以我们互相注视,互相关爱。我给予它生命和热量,它给予我生存的理由。”

“你了解爱。”男孩说道。

“并且了解世界灵魂,因为在沿着宇宙的没有尽头的旅行中,我们经常进行交谈。它对我说,它的最大的问题是,迄今为止,只有矿物和植物才明白万物为一。为此,无需让铁与铜一模一样,让铜与金一模一样。每种物质发挥它的独一无二的功能,万物就会汇成一首和平的交响乐,假如那只写定一切的手在创世的第五天便停下来的话。

“但是还有个第六天。”太阳说道。

“你是个智者,因为你是从远处观察万物。”男孩说道,“但是你不了解爱。假如没有创世的第六天,就不会有人类存在,铜将永远是铜,铅将永远是铅。的确,每种事物都有自己的天命,但是这种天命总有一天将会实现,于是就需要转化成更好的事物,并有了一种新的天命,直至世界灵魂真正地成为唯一之物。”

太阳陷入沉思,决定更加用力地发射出光芒。十分欣赏这场对话的风则更加用力地刮了起来,以免太阳弄瞎了男孩的眼睛。

“为此而有了炼金术,”男孩说道,“目的在于让每个人去寻觅和找到他的财宝,然后他就会希望自己变得要比过去更好。铅将完成它的职能,直至世界不再需要它时为止,这时候它就必定要变成金子。

“这便是炼金术士们所做的事情。他们向世人表明,当我们努力使自己变得比现在更好的时候,我们周围的一切也会变得更好。”

“为什么你说我不了解爱呢?”太阳问道。

“因为爱不是像沙漠那样停滞不动,不是像风那样跑遍世界,也不像你那样从远处观望万物。爱是一种力量,这种力量可以改变和完善世界灵魂。当我第一次深人到世界灵魂中去的时候,我认为它是完美无缺的。但是后来我发现,它乃是所有造物的一种反射,也有自己的冲突和激情。正是我们滋养着世界灵魂,我们所生活的地球变得更好或是更坏,就看我们是变得更好或是更坏。此刻爱的力量就发挥了作用,因为当我们爱的时候,总是希望变得要比原来更好。”

“你想从我这里得到什么呢?”太阳问道。

“帮助我把自己变成风。”男孩回答说。

“自然界认为我是万物中最有智慧的,”太阳说道,“可我不知道如何把你变成风。”

“那么我应该找谁谈呢?”

太阳沉默了片刻。一直在侧耳静听的风要告诉全世界,太阳的智慧是有限的,没有办法应付这个会讲世界语言的男孩。

“你去找写定一切的那只手谈吧。”太阳说道。

风高兴地大声呼叫,比过去任何时候都更加用力地刮了起来。帐篷被从沙地拔起,牲畜挣脱了缚绳。岩石上的人们互相抓紧,以免被抛向远方。

男孩于是转向写定了一切的那只手。男孩没有讲任何话,他感到整个宇宙都沉静了下来,于是他也沉静下来。

一股爱的力量涌出了他的心灵,男孩开始祈祷。这是一种他过去从未做过的祈祷,因为这一祈祷没有使用话语,或是说没有提出请求。男孩没有因为羊群找到了一个牧场而表示感谢,没有为能卖出更多的水晶制品而进行祈求,也没有要求让他遇到的那个女人等待他的归去。在接下来的沉静中,男孩明白了,沙漠、风和太阳也在寻找那只手所写出的预兆,并努力走完自己的道路和理解写在那块简单的绿宝石书板上的东西。男孩知道,那些预兆遍布大地与太空,表面上没有任何目的或意义,而且无论是沙漠、风、太阳还是人,都不知道为什么自己被创造出来。但是那只手作出这一切都有其原因,而且只有它能够创造奇迹,能够把海洋变成沙漠,把人变成风。因为只有它明白,一项宏大的设计将宇宙推向某一个点,在这个点上,六天的创世转化成终极之作。

男孩深入进世界灵魂,看到世界灵魂是上帝灵魂的一部分,接着又看到上帝的灵魂就是他自己的灵魂,因此他也可以创造奇迹。

那一天,西蒙风刮得空前猛烈。在许多许多的世代,阿拉伯人相互讲述着有关一个男孩的神话:他将自己变成了风,几乎摧垮了一座军营,向沙漠里最为显赫的将军的权势发起了挑战。

当西蒙风停息下来时,所有人都朝男孩所在的地方望去。男孩已经不在那里了,而是在军营的另一侧,位于一个在那里站岗的卫兵的身边,这名卫兵几乎被沙土所覆盖。

人们被这一怪事吓坏了,只有两个人露出了微笑:一个是炼金术士,因为他终于找到了自己的真正弟子;另一位是指挥官将军,因为这位弟子理解了上帝的荣耀。

第二天,将军向男孩和炼金术士告别,并派出一支卫队护送他们直至两个人想要到的地方为止。

他们走了整整一天。傍晚时分,一行人来到一座科普特修道院的门前。炼金术士请卫队返回,然后下了马。

“从现在起,你就要独自一人上路了。”炼金术士对男孩说道,“这里距离金字塔只有三个小时的路程。”

“谢谢,”男孩说道,“你教会了我世界语言。”

“我只是让你回忆起你已经知道的东西而已。”炼金术士说道。

炼金术士敲了敲修道院的大门。一位全身穿着黑色衣服的修道士前来开门。他们用科普特语交谈了一会儿,而后炼金术士示意男孩进去。

“我请求他们让我借用一下厨房。”

他们径直来到修道院的厨房。炼金术士升起火,那位修道士拿来了一些铅。炼金术士把铅放进一个铁罐里进行融化,当铅变成液体时,炼金术士从他的袋子里取出那个奇怪的黄色玻璃蛋,从上面刮下一根像头发那样细小的薄片,用蜡封起来,和铅一起放进平底锅里。

两种东西混合后变成了血一样的红色。炼金术士将锅从火上挪开,放到一旁让它冷却。与此同时,他与修道士谈论起部落之间发生的战争。

“大概要持续很长的时间。”他对修道士说道。

修道士显得有些厌烦。不少商队在吉萨己经停留很长时间了,等待着战争的结束。”但是上帝的意愿是会实现的。”修道士说道。

“千真万确。”炼金术士回答说。

平底锅冷却之后,修道士和男孩朝它望去时不禁一阵目眩。铅液已凝固成锅的圆形形状,但却不再是铅,而是成了黄金。

“将来有一天我要不要学会这种本事呢?”男孩问道。

“这是我的天命,而不是你的。”炼金术士回答说,“但我想向你表明,这确实是可能的。”

他们又回到修道院门口。在那里,炼金术士把金盘分成了四块。

“这块给你,”炼金术士边说边把它递给修道士,“因为你对外乡客能够慷慨相待。”

“我得到的报酬超过了我的慷慨。”修道士回答说。

“你永远不要重复这句话,生活可能会听到,下一次就会少给你。”

然后他走近男孩。

“这一块是给你的,以补偿你给指挥官的那些金币。”

男孩刚要说这比他留给指挥官的要多出许多,但却没有讲出口,因为他刚才听到了炼金术士对修道士所说的话。

“这一块是给我的,”炼金术士边说边把它收了起来,“因为我必须穿越沙漠回去,而那里部落之间正在打仗。”

此时他又拿起第四块金子,再次递给了修道士。

“这一块是留给男孩的,如果将来他需要的话。”

“可是我正要去寻找我的财宝,”男孩说道,“我现在离它很近了!

“我坚信你会找到的。”炼金术士说道。

“那为什么要留给我呢?”

“因为你己经两次在旅程中失去了你挣到的钱,一次是被小偷骗走,一次是给了指挥官。我是一个迷信的阿拉伯老人,我相信我们这个地方的各种谚语。有一个谚语说:只发生过一次的事情可能永远不再发生,但是发生过两次的事情一定还会发生第三次。”

他们骑上了各自的马。

“科普特”原意为埃及人,是阿拉伯人于七世纪中叶进人埃及时对埃及

原有居民的称呼。后专指信仰科普特派基督教的人。

“我想给你讲一个关于梦的故事。”炼金术士说道。

男孩把他的马靠近了炼金术士。

“在古代罗马提比略皇帝当政时代,一个心地十分善良的人有两个儿子。一个是军人,入伍时被派往帝国最边远的地区。另一个是诗人,他用美丽的诗句迷住了整个罗马。

“一天夜里,老人做了一个梦,一位天使前来对他说,他的一个儿子的话语将世世代代流传下去,为全世界的人所熟知和反复吟诵。那天夜里老人醒来后感激涕零,因为生活十分慷慨,向他揭示了任何一位父亲知道后都会感到自豪的一件事情。

“不久之后,老人在试图抢救一个行将被芋轮压过的小孩时死去。由于他一生中行为端正,无可指摘,所以就直接去了天堂,并遇到了在他梦中出现的那位天使。

“你一直是个好人,”天使对他说,“活着的时候充满爱心,死得也很尊严。现在我可以实现你的任何愿望。”

“‘生活对我也一直很好,'老人回答说,‘当你出现在我的梦中时,我感到我的所有努力都被证明是正确的。因为我儿子的诗句将在后世的人们之间流传。我为自己提不出任何要求,但是所有的父亲都会为看到自己的儿子享有盛名而感到骄傲。儿子小的时候做父亲的曾养育过他,儿子年轻的时候做父亲的曾教育过他。我希望到遥远的未来去,看看我儿子留下的话语。’”

“天使摸摸老人的肩膀,他们俩便一起被投向一个遥远的未来。他们置身于一个广阔无根的地方,有数以千计的人在讲着一种奇怪的语言。

“老人高兴地哭了。

“‘我知道,我的当诗人的儿子的诗句是精彩和不朽的,'他合着泪对天使说道,‘我想请你告诉我,这些人反复朗诵的是他的哪一首诗?’

“天使亲切地靠近老人,一起在那个广阔无边的地方的长椅之一上坐了下来。

“你的当诗人的儿子的诗句当年在罗马非常流行,'天使说道,‘所有的人都喜欢他的诗句,并以此为娱乐。但是当提比略王朝终结之时,他的诗句也就被人遗忘了。现在人们反复念诵的是你的从军的儿子所说过的话。'

“老人吃惊地望着天使。

“你的儿子在一个偏远的地方服役,后来成为了百夫长。他也是个公正而善良的好人。一天下午,他的一个仆人生了病,眼看就要不行了。你的儿子这时听人说有一个犹太教教士能够治愈百病,于是就一连多日前去寻找这个人。走在路上时,他发现他正在寻找的人就是上帝之子。他遇到了一些被这个教士治愈的病人,学会了这个教士的教义,尽管他身为罗马的百夫长,却改变了自己的信仰。一天上午,他终于来到了这个教士的身边。

“他对教士说他的一个仆人病了。教士准备前往他的家。百夫长是个虔诚之人,当他们周围的人起身的时候,他望着教士眼睛的深处,明白了他面前的正是上帝之子。

“下面就是你儿子所说的话,'天使对老人说道,‘他那个时候对教士所说的话再也没有被人们忘记:主啊,我不配劳您去我的家,但您只要说上一句话,我的仆人就能得救。'”

炼金术士策动了他的马。

“无论做什么,世上的每一个人都总是在世界历史上扮演着主要角色,”他说道,“而人们通常却并不知道这一点。”

男孩微微一笑。过去他从未想过,对一个牧羊人而言,生活竟能如此重要。

“再见。”炼金术士说。

“再见。”男孩回应说。

提比略(前42—公元37):古代罗马第二代皇帝。他以克敌制胜出名,以体恤士卒著称。

男孩在沙漠中行进了两个半小时,竭力要认真地倾听他的心灵所讲的话。心灵将会揭示出准确宝藏之处。

“你的财宝所在之地,也正是你的心灵所在之处。”炼金术士曾这样说过。

但是他的心灵却述说着其他的事情。它很骄傲地讲起一个牧羊人的故事,为了追寻两个夜晚所重复的同一个梦,这个牧羊人离开了他的羊群。它还讲到了天命,讲到了许多追寻天命的人,他们或是寻找远方的土地,或是寻找美丽的女人,为此,他们要面对他们那个时代的具有偏见思想的人。一路上,心灵一直在谈论旅行、发现、书籍和巨大的变化。

当男孩就要开始爬上一座沙丘时——仅仅是在那个时刻——,他的心灵在他耳边低语道:“要留意你将流泪的那个地方,因为那是我的所在之处,你的财宝就在那里。”

男孩开始慢慢向沙丘上方爬去。繁星满天,一轮圆月再度出现在空中。他们在沙漠中已行走了一个月。月光同样地照耀着沙丘,游戏般地投下片片阴影,使沙漠宛如一个波浪翻滚的大海。这使男孩回想起一匹马在沙漠自由驰骋,带给炼金术士一个好预兆的那一天。月亮最终照耀出沙漠的沉寂和寻宝之人的旅程。

几分钟之后,当男孩来到沙丘的顶部时,他的心灵激烈地一跳。在圆月的光芒和沙漠的白色映照下,耸立着的埃及金字塔庄严而壮丽。

男孩跪倒在地,哭泣了起来。他感谢上帝使他相信了自己的天命,让他在某一天遇到了一位圣王,一个水晶店店主,一位英国人,还有一位炼金术士,尤其是遇到了一位沙漠女人,这个女人使他明白,爱永远不会让一个人远离自己的天命。

经历过许多世纪的埃及金字塔俯视着男孩。如果男孩愿意,他现在可以返回绿洲,和法蒂玛在一起,像一个普通的牧羊人那样生活。炼金术士尽管懂得世界语言,知道如何把铅变成金,却依然生活在沙漠里。他无需向任何人展示他的学识和技艺。在追寻天命的路程中,男孩已然学会了他所需要知道的一切,经历了他所梦想要经历的一切。

但他终于来到了他的财宝所在之地,而只有目标实现之时,一件事情才算完成。在那个沙丘上,男孩曾哭泣过。他望了望地面,看到在他的泪水滴落之处有一只金龟子在爬行。在沙漠度过的这段时间里,男孩已经知道,在埃及,金龟子乃是上帝的象征。

又多了一个预兆。男孩在回想起水晶店店主之后便开始在沙丘上挖了起来。无论是谁,哪怕是他一辈子都在堆积石头,也无法在他的后院建造起一座金字塔。

整整一夜,男孩在他选定的地方不断挖掘,却一无所获。金字塔的许多世纪从塔顶静静地注视着他。男孩并没有放弃,他挖呀挖呀,与风展开了抗争,因为风多次把挖出的沙土重又吹回洞穴。男孩的双手先是感到疲劳,随后又受了伤,但他相信自己的心灵。心灵曾告诉他,要在他的泪水滴落之处进行挖掘。

正当他试图把洞中出现的几块石头挖出来的时候,突然听到了一阵脚步声。有几个人来到了他的身边。这些人背对着月亮,使男孩无法看清他们的眼睛和面孔。

“你在这里正干什么呢?”其中的一个人影问道。

男孩没有回答,但却感到了恐惧。现在他马上就要挖到财宝了,所以才心生恐惧。

“我们是逃避部落战争的难民,”另外一个人影说道,“我们需要知道,你在这里藏了什么东西。我们需要钱用。”

“我什么也没藏。”男孩回答说。

但是,其中的一个人还是抓住男孩,把他拖出了洞外。另外一个人开始搜查男孩的钱袋,找到了那块金子。

“他有金子。”那个强盗说道。

月光照在搜查男孩的那个人的脸上,男孩从此人的眼睛中看到了死亡。

“大概他在地下藏有更多的金子。”另外一个人说道。

于是他们便逼迫男孩去挖。男孩继续往下挖,但什么也没有挖到。这伙人开始殴打男孩,一直殴打到天空出现了最初的几缕阳光时才停了下来。男孩的衣服已成为碎片,他感到死亡正在向他逼近。

“假如你难免一死的话,金钱对你又有什么用处呢?钱能够让某个人免于一死的事是很少见的。”炼金术士曾经这样说过。

“我正在寻找一笔财宝!”男孩终于高喊起来。尽管他的嘴被打伤和肿胀,但男孩还是对这伙人说,他曾经两次梦见埃及金字塔附近埋藏着一批财宝。

那位似乎是领头的人许久没有讲话。后来他对他们其中的一个人说道:

“可以把他放开,他没有任何更多的东西了。那块金子大概是偷来的。”

男孩脸朝下倒在了沙地上。那个领头的人的一双眼睛寻找着男孩的眼睛,而男孩的双眼正望着金字塔。

“我们走吧。”领头的人对其他人说。

而后又转向男孩说:

“你不会死的,你会活下去,而且会懂得一个人不能如此愚蠢。两年前,就是在你现在呆着的地方,我也两次做了同一个梦,梦见我应该到西班牙的原野上去,寻找一座倒塌的教堂,牧羊人经常带着他们的羊群在那里过夜,圣器室里生长着一棵埃及榕,如果我从这棵埃及榕的根部挖下去,就一定会找到一批埋藏着的财宝。可是我并不愚蠢,不会仅仅因为一个做过两次的梦而穿越一座沙漠。”

说完他就离去了。

男孩艰难地站立起来,再次朝金字塔望去。金字塔在朝他微笑,而男孩也以微笑回报,心灵充满了幸福。

他找到了财宝。

\chapter{尾声}\label{ch1}

男孩名叫圣地亚哥。暮色儿乎已耍降临时,他来到了那座废弃的小教堂。埃及榕依然生长在圣器室里,透过毁了一半的屋顶依然能看到群星。男孩回忆起,有一次他曾领着羊群来到这里,除了做梦之外那是一个宁静的夜晚。

现在他又来到了这里,不是领着羊群,而是带来了一把铁锹。

他久久地凝视着天空。后来他从褪搭里取出一瓶酒喝了起来。他回忆起在沙漠的那个夜晚,当时他也是望着群星,与炼金术士一起喝酒。他想起了他走过的许多道路,还有上帝为他指出财宝的奇异方式。假如当初他不相信两次做过的同一个梦,就不会遇到那个吉卜赛女人、圣王、小偷以及……“是啊,那是个极长的名单,但是道路已被种种预兆写定,因此我不会走错。”他自言自语道。

男孩不知不觉地睡着了,当他醒来时,太阳已经升起很高了。于是他开始在埃及榕的根部挖掘起来。

“老巫师,”男孩想道,“你无所不知,甚至给我留下了一块金子,好让我能回到这个教堂来。看到我穿着破烂衣衫回去时,那位修道士笑了。难道你就不能让我免受这次罪吗?”

“不能。”男孩听到风声说道,“如果我事先告诉你,你就不会看到金字塔。它们非常壮丽,你不这样认为吗?”

这是炼金术士的声音。男孩微微笑了,接着又继续挖掘。半个小时之后,铁锹碰到了某个坚硬的东西。一个小时之后,男孩眼前出现了一个装满西班牙古金币的箱子。箱子里还有宝石、装饰着红白两种颜色羽毛的金制面具和镶嵌着钻石的石制偶像。这是一次征服的战利品,国家早就把它们遗忘了,而征服者本人也忘记了将其告诉他的子女们。

男孩从褪搭里取出乌陵和土明。他只使用过一次这两块宝石,是某个早晨在一个市场上。生活及其道路总是充满着预兆。

男孩把乌陵和土明收进宝箱里。这也是他的财宝的一部分,因为它们能使他回想起那位他将永远再也无法见到的老圣王。

“生活对追随自己天命之人是慷慨的,”男孩想道。此刻他回想起他必须要到塔里法去,把全部财宝的十分之一送给那位吉卜赛女人。“吉卜赛人是何等地聪明啊,”男孩想道。也许是这样的,因为他们到处旅行。

风又重新刮了起来。是来自非洲的黎凡特风。它没有带来沙漠的气息,也没有带来摩尔人人侵的威胁。相反,它带来了一股他非常熟悉的香味,还有一个吻的响声——这个吻缓缓地、缓缓地而来,直到停在他的双唇上。

男孩微微笑了。这是她第一次这样做。

“我来了,法蒂玛。”他说道。

\end{document}